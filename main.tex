\documentclass[a4paper, 12pt]{extarticle}


\usepackage[table,dvipsnames]{xcolor} % I need to put this here, if not: Option clash for package xcolor error

%================== Required packages ===================

\RequirePackage[french]{babel} %Langue du document
\RequirePackage[utf8]{inputenc} %Caractères spéciaux
\RequirePackage[section]{placeins}%Pour placement de section
\RequirePackage[T1]{fontenc} %Quelques lettres qui sont pas inclus dans UTF-8
\RequirePackage{mathtools} %Paquet pour des équations et symboles mathématiques
% \RequirePackage{siunitx} %Pour écrire avec la notation scientifique (Ex.: \num{2e+9})
\RequirePackage{float} %Pour placement d'images
\RequirePackage{graphicx} %Paquet pour insérer des images
\RequirePackage[justification=centering]{caption} %Pour les légendes centralisées
\RequirePackage{subcaption}
\RequirePackage{wallpaper}
\RequirePackage{nomencl}
\RequirePackage[left=2cm,right=2cm,top=1.5cm,bottom=3.5cm]{geometry} % Margins
\RequirePackage{fancyhdr}





%================== Packages ===================
\usepackage{framed}
\usepackage[normalem]{ulem}
\usepackage{indentfirst}
\usepackage{amsmath,amsthm,amssymb,amsfonts}
\usepackage[nointegrals]{wasysym} % nointegrals prevents wasysym from overwriting integral symbols from LaTeX and amsmath
\usepackage{bbm} % For extended bold and blackboard bold characters
\usepackage[italicdiff]{physics} % italicdiff causes derivatives to be rendered with italic d's instead of upright d's
\usepackage{xparse}
\usepackage{xstring}
% \usepackage{pifont} % For unusual symbols
% \usepackage{mathdots} % For unusual combinations of dots
\usepackage{wrapfig}
\usepackage{lmodern,mathrsfs}
\usepackage[inline,shortlabels]{enumitem}
\usepackage[most]{tcolorbox}
\usepackage{tikz,tikz-3dplot,tikz-cd,tkz-tab,tkz-euclide,pgf,pgfplots}
% \usepackage{comment} % For commenting large blocks of text and math efficiently
% \usepackage{fancyvrb} % For custom verbatim environments
\usepackage{multicol}
\usepackage[bottom,multiple]{footmisc} % Ensures footnotes are at the bottom of the page, and separates footnotes by a comma if they are adjacent
\usepackage[backend=biber,style=numeric]{biblatex}
\usepackage[colorlinks,linkcolor=.,citecolor=blue,urlcolor=violet]{hyperref}
\usepackage[nameinlink]{cleveref} % nameinlink ensures that the entire element is clickable in the pdf, not just the number
\usepackage[explicit]{titlesec}
\usepackage[outputdir=build, newfloat]{minted}
\usepackage{csquotes} % Must be loaded AFTER inputenc
\usepackage{chngcntr}
 % Load packages


\setlist{topsep=2pt,itemsep=2pt,parsep=0pt,partopsep=0pt} %change the format of lists
\pgfplotsset{compat=newest} %use newest version of pgfplots
\counterwithin{figure}{section} % Lier la numérotation des figures à la section


\renewcommand*{\finalnamedelim}{\addcomma\addspace} % Forces authors' names to be separated by comma, instead of "and"

\addbibresource{bibliography.bib} % Add bibliography file

\newcommand{\remind}[1]{\textcolor{red}{\textbf{#1}}} % To remind me of unfinished work to fix later
\newcommand{\hide}[1]{} % To hide large blocks of code without using % symbols

% Same as \href, but the text appears in typewriter font and in a custom color
\newcommand{\Href}[3][red!50!black]{\href{#2}{\textcolor{#1}{\texttt{#3}}}}

\newcommand{\ep}{\varepsilon}
\newcommand{\vp}{\varphi}
\newcommand{\lam}{\lambda}
\newcommand{\Lam}{\Lambda}
\DeclareDocumentCommand\ip{ l m }{\braces#1{\langle}{\rangle}{#2}} % Inner product ⟨x,y⟩ (but only one argument is taken, so \ip{x,y} renders as ⟨x,y⟩)
\DeclareDocumentCommand\floor{ l m }{\braces#1{\lfloor}{\rfloor}{#2}} % Floor function ⌊x⌋
\DeclareDocumentCommand\ceil{ l m }{\braces#1{\lceil}{\rceil}{#2}} % Ceiling function ⌈x⌉

% Shortcuts for blackboard bold letters, e.g. \A outputs \mathbb{A}
\def\do#1{\csdef{#1}{\mathbb{#1}}}
\docsvlist{A,B,C,D,E,F,G,I,J,K,M,N,Q,R,T,U,V,W,X,Y,Z}
% \H is already defined as a 1-argument command, it places a double acute accent (hungarumlaut) on a character, e.g. \H{o} yields ő
% \L is already defined as the uppercase Ł (L with stroke)
% \O is already defined as the uppercase Ø (O with stroke)
% \P is already defined as the pilcrow ¶ (paragraph mark)
% \S is already defined as the section sign §

% Shortcuts for calligraphic letters, e.g. \As outputs \mathcal{A}
\def\do#1{\csdef{#1s}{\mathcal{#1}}}
\docsvlist{A,B,C,D,E,F,G,H,I,J,K,L,M,N,O,P,Q,R,S,T,U,V,W,X,Y,Z}

% Shortcuts for letters with a bar on top, e.g. \Abar outputs \overline{A}
\def\do#1{\csdef{#1bar}{\overline{#1}}}
\docsvlist{a,b,c,d,e,f,g,i,j,k,l,m,n,o,p,q,r,s,t,u,v,w,x,y,z,A,B,C,D,E,F,G,H,I,J,K,L,M,N,O,P,Q,R,S,T,U,V,W,X,Y,Z}
% \hbar is already defined as the symbol ℏ (reduced Planck constant)

% Shortcuts for boldface letters, e.g. \Ab outputs \textbf{A}
\def\do#1{\csdef{#1b}{\textbf{#1}}}
\docsvlist{a,b,c,d,e,f,g,h,i,j,k,l,m,n,o,q,r,t,u,w,x,y,z,A,B,C,D,E,F,G,H,I,J,K,L,M,N,O,P,Q,R,S,T,U,V,W,X,Y,Z}
% \pb is already defined (by the physics package) as a 2-argument command, denoting the anticommutator or Poisson bracket, e.g. \pb{A,B} yields {A,B}
% \sb is already defined in the LaTeX kernel. This is a fundamental LaTeX command, DO NOT overwrite it!
% \vb is already defined (by the physics package) as a 1-argument command, for boldface text, e.g. \vb{A} yields \textbf{A}

% Shortcuts for letters with a tilde on top, e.g. \Atil outputs \widetilde{A}
\def\do#1{\csdef{#1til}{\widetilde{#1}}}
\docsvlist{a,b,c,d,e,f,g,h,i,j,k,l,m,n,o,p,q,r,s,t,u,v,w,x,y,z,A,B,C,D,E,F,G,H,I,J,K,L,M,N,O,P,Q,R,S,T,U,V,W,X,Y,Z}

\newcommand{\tm}{^{\mathsf{T}}}     % Transpose
\newcommand{\hm}{^{\mathsf{H}}}     % Conjugate transpose (Hermitian conjugate)
\newcommand{\itm}{^{-\mathsf{T}}}   % Inverse transpose
\newcommand{\ihm}{^{-\mathsf{H}}}   % Inverse conjugate transpose (Inverse Hermitian conjugate)
\newcommand{\ex}{\textbf{e}_x}
\newcommand{\ey}{\textbf{e}_y}
\newcommand{\ez}{\textbf{e}_z}
\newcommand{\Aint}{A^\circ}
\newcommand{\Bint}{B^\circ}
\newcommand{\limk}{\lim_{k\to\infty}}
\newcommand{\limm}{\lim_{m\to\infty}}
\newcommand{\limn}{\lim_{n\to\infty}}
\newcommand{\limx}[1][a]{\lim_{x\to#1}}
\newcommand{\limz}[1][{z_0}]{\lim_{z\to#1}}
\newcommand{\liminfm}{\liminf_{m\to\infty}}
\newcommand{\limsupm}{\limsup_{m\to\infty}}
\newcommand{\liminfn}{\liminf_{n\to\infty}}
\newcommand{\limsupn}{\limsup_{n\to\infty}}
\newcommand{\sumkn}{\sum_{k=1}^n}
\newcommand{\sumk}[1][1]{\sum_{k=#1}^\infty}
\newcommand{\summ}[1][1]{\sum_{m=#1}^\infty}
\newcommand{\sumn}[1][1]{\sum_{n=#1}^\infty}
\newcommand{\emp}{\varnothing}
\newcommand{\exc}{\backslash}
\newcommand{\sub}{\subseteq}
\newcommand{\sups}{\supseteq}
\newcommand{\capp}{\bigcap}
\newcommand{\cupp}{\bigcup}
\newcommand{\kupp}{\bigsqcup}
\newcommand{\cappkn}{\bigcap_{k=1}^n}
\newcommand{\cuppkn}{\bigcup_{k=1}^n}
\newcommand{\kuppkn}{\bigsqcup_{k=1}^n}
\newcommand{\cappk}[1][1]{\bigcap_{k=#1}^\infty}
\newcommand{\cuppk}[1][1]{\bigcup_{k=#1}^\infty}
\newcommand{\cappm}[1][1]{\bigcap_{m=#1}^\infty}
\newcommand{\cuppm}[1][1]{\bigcup_{m=#1}^\infty}
\newcommand{\cappn}[1][1]{\bigcap_{n=#1}^\infty}
\newcommand{\cuppn}[1][1]{\bigcup_{n=#1}^\infty}
\newcommand{\kuppk}[1][1]{\bigsqcup_{k=#1}^\infty}
\newcommand{\kuppm}[1][1]{\bigsqcup_{m=#1}^\infty}
\newcommand{\kuppn}[1][1]{\bigsqcup_{n=#1}^\infty}
\newcommand{\cappa}{\bigcap_{\alpha\in I}}
\newcommand{\cuppa}{\bigcup_{\alpha\in I}}
\newcommand{\kuppa}{\bigsqcup_{\alpha\in I}}
\newcommand{\dx}{\,dx}
\newcommand{\dy}{\,dy}
\newcommand{\dt}{\,dt}
\newcommand{\dmu}{\,d\mu}
\newcommand{\dnu}{\,d\nu}
\DeclareMathOperator{\glb}{\text{glb}}
\DeclareMathOperator{\lub}{\text{lub}}
\newcommand{\xh}{\widehat{x}}
\newcommand{\yh}{\widehat{y}}
\newcommand{\zh}{\widehat{z}}
\newcommand{\<}{\langle}
\renewcommand{\>}{\rangle}

% Shortcuts for inverse hyperbolic functions (and other operators with the same structure)
\def\do#1{\csdef{#1}{\trigbraces{\operatorname{#1}}}}
\docsvlist{
    asinh,acosh,atanh,acoth,asech,acsch,
    arsinh,arcosh,artanh,arcoth,arsech,arcsch,
    arcsinh,arccosh,arctanh,arccoth,arcsech,arccsch,
    sen,tg,cth,senh,tgh,ctgh,
    Re,Im,arg,Arg,im,ker
}

% \spn has to be defined separately as the syntax "spn" is different from the output "span"
% \span is already defined in the LaTeX kernel. This is a fundamental LaTeX command, DO NOT overwrite it!
\newcommand{\spn}{\trigbraces{\operatorname{span}}}

\makeatletter
% Redefining the commands \iff (given by LaTeX), \implies and \impliedby (given by amsmath)
% Math mode is automatically enforced, starred version makes the arrows shorter
\renewcommand{\impliedby}{\@ifstar{\ensuremath{\Longleftarrow}}{\ensuremath{\Leftarrow}}} % Corresponding Unicode character: U+21D0 ⇐
\renewcommand{\implies}{\@ifstar{\ensuremath{\Longrightarrow}}{\ensuremath{\Rightarrow}}} % Corresponding Unicode character: U+21D2 ⇒
\renewcommand{\iff}{\@ifstar{\ensuremath{\Longleftrightarrow}}{\ensuremath{\Leftrightarrow}}} % Corresponding Unicode character: U+21D4 ⇔
\makeatother

\input{letterfonts}


\newtheoremstyle{mystyle}{}{}{}{}{\sffamily\bfseries}{:}{ }{}
\makeatletter
\renewenvironment{proof}[1][\proofname] {\par\pushQED{\qed}
{\normalfont\sffamily\bfseries\topsep6\p@\@plus6\p@\relax #1\@addpunct{:} }}{\popQED\endtrivlist\@endpefalse}
\makeatother
\renewcommand{\qedsymbol}{\coolqed{0.32}} % Implements the new QED symbol
\theoremstyle{mystyle}{\newtheorem*{remark}{Remarque}}
\theoremstyle{mystyle}{\newtheorem*{remarks}{Remarques}}
\theoremstyle{mystyle}{\newtheorem*{example}{Exemple}}
\theoremstyle{mystyle}{\newtheorem*{examples}{Exemples}}
\theoremstyle{definition}{\newtheorem*{exercise}{Exercice}}

% Warning environment
\newtheoremstyle{warn}{}{}{}{}{\normalfont}{}{ }{}
\theoremstyle{warn}
\newtheorem*{warning}{\warningsign{0.2}\relax}

% Symbol for the warning environment, designed to be easily scalable
\newcommand{\warningsign}[1]{
    \tikz[scale=#1,every node/.style={transform shape}]{
        \draw[-,line width={#1*0.8mm},red,fill=yellow,rounded corners={#1*2.5mm}] (0,0)--(1,{-sqrt(3)})--(-1,{-sqrt(3)})--cycle;
        \node at (0,-1) {\fontsize{48}{60}\selectfont\bfseries!};
}}

\newcommand{\coolqed}[1]{\includegraphics[width=#1cm]{sunglasses_emoji.png}} % QED symbol

% verbbox environment, for showing verbatim text next to code output (for package documentation and user learning purposes)
\NewTCBListing{verbbox}{ !O{} }{boxrule=1pt,sidebyside,skin=bicolor,colback=gray!10,colbacklower=white,valign=center,top=2pt,bottom=2pt,left=2pt,right=2pt,#1} % Last argument allows more tcolorbox options to be added

\NewDocumentCommand{\solidball}{ s O{} m O{white} m }{
\tikz[scale=#5,every node/.style={transform shape}]{
    \shade[ball color=#3] (0,0) circle (0.5); %solid ball with no label
    \IfBooleanF{#1}{
        \clip (0,0) circle (0.25);
        \shade[ball color=#4] (0,0) circle (0.5);
    }
    \node[font=\sffamily\bfseries\selectfont] at (0,0) {#2}; % Label
}
}

\NewDocumentCommand{\stripedball}{ s O{} m O{white} m }{
\tikz[scale=#5,every node/.style={transform shape}]{
    \shade[ball color=#4] (0,0) circle (0.5);
    \clip (-0.5,-0.35) rectangle (0.5,0.35);
    \shade[ball color=#3] (0,0) circle (0.5);
    \IfBooleanF{#1}{
        \clip (0,0) circle (0.25);
        \shade[ball color=#4] (0,0) circle (0.5);
    }
    \node[font=\sffamily\bfseries\selectfont] at (0,0) {#2}; % Label
}
}

% Official colors for Aramith Tournament pool balls
% Colors taken from https://www.aramith.com/story-behind-aramith-tournament-black-colours
\definecolor{aramith_color_0}{HTML}{FFFFDF} % Cue ball and secondary color for all balls
\definecolor{aramith_color_1}{HTML}{FFD501} % 1 and 9
\definecolor{aramith_color_2}{HTML}{013CB1} % 2 and 10
\definecolor{aramith_color_3}{HTML}{E71C01} % 3 and 11
\definecolor{aramith_color_4}{HTML}{4F029C} % 4 and 12
\definecolor{aramith_color_5}{HTML}{FA4D00} % 5 and 13
\definecolor{aramith_color_6}{HTML}{0E5D01} % 6 and 14
\definecolor{aramith_color_7}{HTML}{6D071A} % 7 and 15
\definecolor{aramith_color_8}{HTML}{000000} % 8 (black)


\makeatletter
% Adapted from https://tex.stackexchange.com/a/61600
\csdef{aramith_pool_ball@0}#1{\solidball*{aramith_color_0}{#1}}                         % Cue ball
\csdef{aramith_pool_ball@1}#1{\solidball[1]{aramith_color_1}[aramith_color_0]{#1}}      % Ball 1
\csdef{aramith_pool_ball@2}#1{\solidball[2]{aramith_color_2}[aramith_color_0]{#1}}      % Ball 2
\csdef{aramith_pool_ball@3}#1{\solidball[3]{aramith_color_3}[aramith_color_0]{#1}}      % Ball 3
\csdef{aramith_pool_ball@4}#1{\solidball[4]{aramith_color_4}[aramith_color_0]{#1}}      % Ball 4
\csdef{aramith_pool_ball@5}#1{\solidball[5]{aramith_color_5}[aramith_color_0]{#1}}      % Ball 5
\csdef{aramith_pool_ball@6}#1{\solidball[6]{aramith_color_6}[aramith_color_0]{#1}}      % Ball 6
\csdef{aramith_pool_ball@7}#1{\solidball[7]{aramith_color_7}[aramith_color_0]{#1}}      % Ball 7
\csdef{aramith_pool_ball@8}#1{\solidball[8]{aramith_color_8}[aramith_color_0]{#1}}      % Ball 8 (black)
\csdef{aramith_pool_ball@9}#1{\stripedball[9]{aramith_color_1}[aramith_color_0]{#1}}    % Ball 9
\csdef{aramith_pool_ball@10}#1{\stripedball[10]{aramith_color_2}[aramith_color_0]{#1}}  % Ball 10
\csdef{aramith_pool_ball@11}#1{\stripedball[11]{aramith_color_3}[aramith_color_0]{#1}}  % Ball 11
\csdef{aramith_pool_ball@12}#1{\stripedball[12]{aramith_color_4}[aramith_color_0]{#1}}  % Ball 12
\csdef{aramith_pool_ball@13}#1{\stripedball[13]{aramith_color_5}[aramith_color_0]{#1}}  % Ball 13
\csdef{aramith_pool_ball@14}#1{\stripedball[14]{aramith_color_6}[aramith_color_0]{#1}}  % Ball 14
\csdef{aramith_pool_ball@15}#1{\stripedball[15]{aramith_color_7}[aramith_color_0]{#1}}  % Ball 15
\NewDocumentCommand{\poolball}{ o m m }{
    \pgfmathparse{Mod(#2,16)} % Argument #2 mod 16  as a floating-point number, e.g. 4.0
    \pgfmathtruncatemacro{\argumentmodulosixteen}{\pgfmathresult} % Convert to integer, e.g. 4.0 to 4
    \ifnum\argumentmodulosixteen=0
        \solidball*{aramith_color_0}{#3}
    \else
        \ifnum\argumentmodulosixteen<9
            \solidball[\IfNoValueTF{#1}{#2}{#1}]{aramith_color_\argumentmodulosixteen}[aramith_color_0]{#3} % Solid ball of the appropriate color, and the appropriate number (if the optional argument is not specified), otherwise the optional argument
        \else
            \pgfmathparse{\argumentmodulosixteen-8} % Argument #2 mod 8  as a floating-point number, e.g. 3.0 (this will only be computed if #2≥9)
            \pgfmathtruncatemacro{\argumentmoduloeight}{\pgfmathresult} % Convert to integer, e.g. 3.0 to 3
            \stripedball[\IfNoValueTF{#1}{#2}{#1}]{aramith_color_\argumentmoduloeight}[aramith_color_0]{#3} % Striped ball of the appropriate color, and the appropriate number (if the optional argument is not specified), otherwise the optional argument
        \fi
    \fi}
\makeatother

\makeatletter
% \fsize stores the current font size but is expandable (and can be called later without using \makeatletter and \makeatother)
\def\fsize{\dimexpr\f@size pt\relax}
\makeatother

\makeatletter
% Adapted from https://tex.stackexchange.com/a/19700
\def\my@vector #1,#2\@eolst{
    \ifx\relax#2\relax
        #1
    \else
        #1\my@delim
        \my@vector #2\@eolst
    \fi}
\newcommand\vcstring[2][\\]{% Converts comma-separated string to #1-separated string
    \def\my@delim{#1}
        \my@vector #2,\relax\noexpand\@eolst}
\newcommand\cvc[2][p]{% Converts comma-separated string to column vector, optional argument defines matrix brackets
    \def\my@delim{\\}
        \begin{#1matrix} % Empty argument also possible
            \my@vector #2,\relax\noexpand\@eolst
        \end{#1matrix}}
\newcommand\rvc[2][p]{% Converts comma-separated string to row vector, optional argument defines matrix brackets
    \def\my@delim{&}
        \begin{#1matrix} % Empty argument also possible
            \my@vector #2,\relax\noexpand\@eolst
        \end{#1matrix}}
% Matrix environment with variable number of arguments. Adapted from https://davidyat.es/2016/07/27/writing-a-latex-macro-that-takes-a-variable-number-of-arguments/
\newcommand{\mat}[2][p]{
    \def\matrixenvironment{#1matrix} % Specifying the matrix brackets, this has to be done beforehand as '#1' changes under \passtonextarg
    \def\my@delim{&}
        \begin{\matrixenvironment} % Begin matrix environment
            \my@vector #2,\relax\noexpand\@eolst
            \@ifnextchar\bgroup{\passtonextarg}{\end{\matrixenvironment}}% % Pass to next argument (if any), otherwise end matrix environment
}
\newcommand{\passtonextarg}[1]{\\ \my@vector #1,\relax\noexpand\@eolst
    \@ifnextchar\bgroup{\passtonextarg}{\end{\matrixenvironment}}% Passing to next argument
}
\makeatother

\definecolor{tcol_DEF}{HTML}{E40125} % Color for Definition
\definecolor{tcol_PRP}{HTML}{EB8407} % Color for Proposition
\definecolor{tcol_LEM}{HTML}{05C4D9} % Color for Lemma
\definecolor{tcol_THM}{HTML}{1346E4} % Color for Theorem
\definecolor{tcol_COR}{HTML}{7904C2} % Color for Corollary
\definecolor{tcol_REM}{HTML}{18B640} % Color for Remark
\definecolor{tcol_PRF}{HTML}{5A76B2} % Color for Proof
\definecolor{tcol_EXA}{HTML}{21340A} % Color for Example
\definecolor{tcol_CNT1}{HTML}{72E094} % First color for Contents
\definecolor{tcol_CNT2}{HTML}{24E2D6} % Second color for Contents
\definecolor{tcol_CNV1}{HTML}{AF7AC5} % First color for Conventions
\definecolor{tcol_CNV2}{HTML}{C3B1E1} % First color for Conventions

\tcbset{
tbox_DEF_style/.style={enhanced jigsaw,
    colback=tcol_DEF!10,colframe=tcol_DEF!80!black,,
    fonttitle=\sffamily\bfseries,
    separator sign=.,label separator={},
    sharp corners,top=2pt,bottom=2pt,left=2pt,right=2pt,
    before skip=10pt,after skip=10pt,breakable,
},
tbox_PRP_style/.style={enhanced jigsaw,
    colback=tcol_PRP!10,colframe=tcol_PRP!80!black,
    fonttitle=\sffamily\bfseries,
    attach boxed title to top left={yshift=-\tcboxedtitleheight},
    boxed title style={
        boxrule=0pt,boxsep=2.5pt,
        colback=tcol_PRP!80!black,colframe=tcol_PRP!80!black,
        sharp corners=uphill
    },
    separator sign=.,label separator={},
    top=\tcboxedtitleheight,bottom=2pt,left=2pt,right=2pt,
    before skip=10pt,after skip=10pt,drop fuzzy shadow,breakable
},
tbox_THM_style/.style={enhanced jigsaw,
    colback=tcol_THM!10,colframe=tcol_THM!80!black,
    fonttitle=\sffamily\bfseries,coltitle=black,
    attach boxed title to top left={xshift=10pt,yshift=-\tcboxedtitleheight/2},
    boxed title style={
        colback=tcol_THM!10,colframe=tcol_THM!80!black,height=16pt,bean arc
    },
    separator sign=.,label separator={},
    sharp corners,top=6pt,bottom=2pt,left=2pt,right=2pt,
    before skip=10pt,after skip=10pt,breakable
},
tbox_LEM_style/.style={enhanced jigsaw,
    colback=tcol_LEM!10,colframe=tcol_LEM!80!black,
    boxrule=0pt,
    fonttitle=\sffamily\bfseries,
    attach boxed title to top left={yshift=-\tcboxedtitleheight},
    boxed title style={
        boxrule=0pt,boxsep=2pt,
        colback=tcol_LEM!80!black,colframe=tcol_LEM!80!black,
        interior code={\fill[tcol_LEM!80!black] (interior.north west)--(interior.south west)--([xshift=-2mm]interior.south east)--([xshift=2mm]interior.north east)--cycle;
    }},
    separator sign=.,label separator={},
    frame hidden,borderline north={1pt}{0pt}{tcol_LEM!80!black},
    before upper={\hspace{\tcboxedtitlewidth}},
    sharp corners,top=2pt,bottom=2pt,left=5pt,right=5pt,
    before skip=10pt,after skip=10pt,breakable
},
tbox_COR_style/.style={enhanced jigsaw,
    colback=tcol_COR!10,colframe=tcol_COR!80!black,
    boxrule=0pt,
    fonttitle=\sffamily\bfseries,coltitle=black,
    separator sign={},label separator={},
    description font=\normalfont\sffamily,
    description delimiters={(}{)},
    attach title to upper,after title={:\ },
    frame hidden,borderline west={2pt}{0pt}{tcol_COR},
    sharp corners,top=2pt,bottom=2pt,left=5pt,right=5pt,
    before skip=10pt,after skip=10pt,breakable
},
}

\newtcbtheorem[number within=section,
    crefname={\color{tcol_DEF!50!black} définition}{\color{tcol_DEF!50!black} définitions},
    Crefname={\color{tcol_DEF!50!black} Définition}{\color{tcol_DEF!50!black} Définitions}
    ]{definition}{Définition}{tbox_DEF_style}{}
\newtcbtheorem[number within=section,
% \newtcbtheorem[use counter from=definition,
    crefname={\color{tcol_PRP!50!black} proposition}{\color{tcol_PRP!50!black} propositions},
    Crefname={\color{tcol_PRP!50!black} Proposition}{\color{tcol_PRP!50!black} Propositions}
    ]{proposition}{Proposition}{tbox_PRP_style}{}
% \newtcbtheorem[use counter from=definition,
\newtcbtheorem[number within=section,
    crefname={\color{tcol_THM!50!black} théorème}{\color{tcol_THM!50!black} théorèmes},
    Crefname={\color{tcol_THM!50!black} Théorème}{\color{tcol_THM!50!black} Théorèmes}
    ]{theorem}{Théorème}{tbox_THM_style}{}
\newtcbtheorem[number within=section,
% \newtcbtheorem[use counter from=definition,
    crefname={\color{tcol_LEM!50!black} lemme}{\color{tcol_LEM!50!black} lemmes},
    Crefname={\color{tcol_LEM!50!black} Lemme}{\color{tcol_LEM!50!black} Lemmes}
    ]{lemma}{Lemme}{tbox_LEM_style}{}
\newtcbtheorem[number within=section,
% \newtcbtheorem[use counter from=definition,
    crefname={\color{tcol_COR!50!black} corollaire}{\color{tcol_COR!50!black} corollaires},
    Crefname={\color{tcol_COR!50!black} Corollaire}{\color{tcol_COR!50!black} Corollaires}
    ]{corollary}{Corollaire}{tbox_COR_style}{}

\makeatletter
\@namedef{tcolorboxshape@filingbox@ul}#1#2#3{
    (frame.south west)--(title.north west)--([xshift=-\dimexpr#1\relax]title.north east) to[out=0,in=180] ([xshift=\dimexpr#2\relax,yshift=\dimexpr#3\relax]title.south east)--(frame.north east)--(frame.south east)--cycle
}
\@namedef{tcolorboxshape@filingbox@uc}#1#2#3{
    (frame.south west)--(frame.north west)--([xshift=-\dimexpr#2\relax,yshift=\dimexpr#3\relax]title.south west) to[out=0,in=180] ([xshift=\dimexpr#1\relax]title.north west)--([xshift=-\dimexpr#1\relax]title.north east) to[out=0,in=180] ([xshift=\dimexpr#2\relax,yshift=\dimexpr#3\relax]title.south east)--(frame.north east)--(frame.south east)--cycle
}
\@namedef{tcolorboxshape@filingbox@ur}#1#2#3{
    (frame.south east)--(title.north east)--([xshift=\dimexpr#1\relax]title.north west) to[out=180,in=0] ([xshift=-\dimexpr#2\relax,yshift=\dimexpr#3\relax]title.south west)--(frame.north west)--(frame.south west)--cycle
}
\@namedef{tcolorboxshape@filingbox@dl}#1#2#3{
    (frame.north west)--(title.south west)--([xshift=-\dimexpr#1\relax]title.south east) to[out=0,in=180] ([xshift=\dimexpr#2\relax,yshift=-\dimexpr#3\relax]title.north east)--(frame.south east)--(frame.north east)--cycle
}
\@namedef{tcolorboxshape@filingbox@dc}#1#2#3{
    (frame.north west)--(frame.south west)--([xshift=-\dimexpr#2\relax,yshift=-\dimexpr#3\relax]title.north west) to[out=0,in=180] ([xshift=\dimexpr#1\relax]title.south west)--([xshift=-\dimexpr#1\relax]title.south east) to[out=0,in=180] ([xshift=\dimexpr#2\relax,yshift=-\dimexpr#3\relax]title.north east)--(frame.south east)--(frame.north east)--cycle
}
\@namedef{tcolorboxshape@filingbox@dr}#1#2#3{
    (frame.north east)--(title.south east)--([xshift=\dimexpr#1\relax]title.south west) to[out=180,in=0] ([xshift=-\dimexpr#2\relax,yshift=-\dimexpr#3\relax]title.north west)--(frame.south west)--(frame.north west)--cycle
}
\@namedef{tcolorboxshape@railingbox@ul}#1#2#3{
    (frame.south west)--(title.north west)--([xshift=-\dimexpr#1\relax]title.north east)--([xshift=\dimexpr#2\relax,yshift=\dimexpr#3\relax]title.south east)--(frame.north east)--(frame.south east)--cycle
}
\@namedef{tcolorboxshape@railingbox@uc}#1#2#3{
    (frame.south west)--(frame.north west)--([xshift=-\dimexpr#2\relax,yshift=\dimexpr#3\relax]title.south west)--([xshift=\dimexpr#1\relax]title.north west)--([xshift=-\dimexpr#1\relax]title.north east)--([xshift=\dimexpr#2\relax,yshift=\dimexpr#3\relax]title.south east)--(frame.north east)--(frame.south east)--cycle
}
\@namedef{tcolorboxshape@railingbox@ur}#1#2#3{
    (frame.south east)--(title.north east)--([xshift=\dimexpr#1\relax]title.north west)--([xshift=-\dimexpr#2\relax,yshift=\dimexpr#3\relax]title.south west)--(frame.north west)--(frame.south west)--cycle
}
\@namedef{tcolorboxshape@railingbox@dl}#1#2#3{
    (frame.north west)--(title.south west)--([xshift=-\dimexpr#1\relax]title.south east)--([xshift=\dimexpr#2\relax,yshift=-\dimexpr#3\relax]title.north east)--(frame.south east)--(frame.north east)--cycle
}
\@namedef{tcolorboxshape@railingbox@dc}#1#2#3{
    (frame.north west)--(frame.south west)--([xshift=-\dimexpr#2\relax,yshift=-\dimexpr#3\relax]title.north west)--([xshift=\dimexpr#1\relax]title.south west)--([xshift=-\dimexpr#1\relax]title.south east)--([xshift=\dimexpr#2\relax,yshift=-\dimexpr#3\relax]title.north east)--(frame.south east)--(frame.north east)--cycle
}
\@namedef{tcolorboxshape@railingbox@dr}#1#2#3{
    (frame.north east)--(title.south east)--([xshift=\dimexpr#1\relax]title.south west)--([xshift=-\dimexpr#2\relax,yshift=-\dimexpr#3\relax]title.north west)--(frame.south west)--(frame.north west)--cycle
}
\newcommand{\TColorBoxShape}[2]{\expandafter\ifx\csname tcolorboxshape@#1@#2\endcsname\relax
\expandafter\@gobble\else
\csname tcolorboxshape@#1@#2\expandafter\endcsname
\fi}
\makeatother

\tcbset{ % Styles for filingbox, railingbox and flagbox environments
% Adapted from https://tex.stackexchange.com/questions/587912/tcolorbox-custom-title-box-style
filingstyle/ul/.style 2 args={
    attach boxed title to top left={yshift=-2mm},
    boxed title style={empty,top=0mm,bottom=1mm,left=1mm,right=0mm},
    interior code={
        \path[fill=#1,rounded corners] \TColorBoxShape{filingbox}{ul}{9pt}{18pt}{6pt};
    },
    frame code={
        \path[draw=#2,line width=0.5mm,rounded corners] \TColorBoxShape{filingbox}{ul}{9pt}{18pt}{6pt};
    }},
filingstyle/uc/.style 2 args={
    attach boxed title to top center={yshift=-2mm},
    boxed title style={empty,top=0mm,bottom=1mm,left=0mm,right=0mm},
    interior code={
        \path[fill=#1,rounded corners] \TColorBoxShape{filingbox}{uc}{9pt}{18pt}{6pt};
    },
    frame code={
        \path[draw=#2,line width=0.5mm,rounded corners] \TColorBoxShape{filingbox}{uc}{9pt}{18pt}{6pt};
    }},
filingstyle/ur/.style 2 args={
    attach boxed title to top right={yshift=-2mm},
    boxed title style={empty,top=0mm,bottom=1mm,left=0mm,right=1mm},
    interior code={
        \path[fill=#1,rounded corners] \TColorBoxShape{filingbox}{ur}{9pt}{18pt}{6pt};
    },
    frame code={
        \path[draw=#2,line width=0.5mm,rounded corners] \TColorBoxShape{filingbox}{ur}{9pt}{18pt}{6pt};
    }},
filingstyle/dl/.style 2 args={
    attach boxed title to bottom left={yshift=2mm},
    boxed title style={empty,top=1mm,bottom=0mm,left=1mm,right=0mm},
    interior code={
        \path[fill=#1,rounded corners] \TColorBoxShape{filingbox}{dl}{9pt}{18pt}{6pt};
    },
    frame code={
        \path[draw=#2,line width=0.5mm,rounded corners] \TColorBoxShape{filingbox}{dl}{9pt}{18pt}{6pt};
    }},
filingstyle/dc/.style 2 args={
    attach boxed title to bottom center={yshift=2mm},
    boxed title style={empty,top=1mm,bottom=0mm,left=0mm,right=0mm},
    interior code={
        \path[fill=#1,rounded corners] \TColorBoxShape{filingbox}{dc}{9pt}{18pt}{6pt};
    },
    frame code={
        \path[draw=#2,line width=0.5mm,rounded corners] \TColorBoxShape{filingbox}{dc}{9pt}{18pt}{6pt};
    }},
filingstyle/dr/.style 2 args={
    attach boxed title to bottom right={yshift=2mm},
    boxed title style={empty,top=1mm,bottom=0mm,left=0mm,right=1mm},
    interior code={
        \path[fill=#1,rounded corners] \TColorBoxShape{filingbox}{dr}{9pt}{18pt}{6pt};
    },
    frame code={
        \path[draw=#2,line width=0.5mm,rounded corners] \TColorBoxShape{filingbox}{dr}{9pt}{18pt}{6pt};
    }},
railingstyle/ul/.style 2 args={
    attach boxed title to top left={yshift=-2mm},
    boxed title style={empty,top=0mm,bottom=1mm,left=1mm,right=0mm},
    interior code={
        \path[fill=#1] \TColorBoxShape{railingbox}{ul}{3pt}{12pt}{6pt};
    },
    frame code={
        \path[draw=#2,line width=0.5mm] \TColorBoxShape{railingbox}{ul}{3pt}{12pt}{6pt};
    }},
railingstyle/uc/.style 2 args={
    attach boxed title to top center={yshift=-2mm},
    boxed title style={empty,top=0mm,bottom=1mm,left=0mm,right=0mm},
    interior code={
        \path[fill=#1] \TColorBoxShape{railingbox}{uc}{3pt}{12pt}{6pt};
    },
    frame code={
        \path[draw=#2,line width=0.5mm] \TColorBoxShape{railingbox}{uc}{3pt}{12pt}{6pt};
    }},
railingstyle/ur/.style 2 args={
    attach boxed title to top right={yshift=-2mm},
    boxed title style={empty,top=0mm,bottom=1mm,left=0mm,right=1mm},
    interior code={
        \path[fill=#1] \TColorBoxShape{railingbox}{ur}{3pt}{12pt}{6pt};
    },
    frame code={
        \path[draw=#2,line width=0.5mm] \TColorBoxShape{railingbox}{ur}{3pt}{12pt}{6pt};
    }},
railingstyle/dl/.style 2 args={
    attach boxed title to bottom left={yshift=2mm},
    boxed title style={empty,top=1mm,bottom=0mm,left=1mm,right=0mm},
    interior code={
        \path[fill=#1] \TColorBoxShape{railingbox}{dl}{3pt}{12pt}{6pt};
    },
    frame code={
        \path[draw=#2,line width=0.5mm] \TColorBoxShape{railingbox}{dl}{3pt}{12pt}{6pt};
    }},
railingstyle/dc/.style 2 args={
    attach boxed title to bottom center={yshift=2mm},
    boxed title style={empty,top=1mm,bottom=0mm,left=0mm,right=0mm},
    interior code={
        \path[fill=#1] \TColorBoxShape{railingbox}{dc}{3pt}{12pt}{6pt};
    },
    frame code={
        \path[draw=#2,line width=0.5mm] \TColorBoxShape{railingbox}{dc}{3pt}{12pt}{6pt};
    }},
railingstyle/dr/.style 2 args={
    attach boxed title to bottom right={yshift=2mm},
    boxed title style={empty,top=1mm,bottom=0mm,left=0mm,right=1mm},
    interior code={
        \path[fill=#1] \TColorBoxShape{railingbox}{dr}{3pt}{12pt}{6pt};
    },
    frame code={
        \path[draw=#2,line width=0.5mm] \TColorBoxShape{railingbox}{dr}{3pt}{12pt}{6pt};
    }},
flagstyle/ul/.style 2 args={
    interior hidden,frame hidden,colbacktitle=#1,
    borderline west={1pt}{0pt}{#2},
    attach boxed title to top left={yshift=-8pt,yshifttext=-8pt},
    boxed title style={boxsep=3pt,boxrule=1pt,colframe=#2,sharp corners,left=4pt,right=4pt},
    bottom=0mm
    },
flagstyle/ur/.style 2 args={
    interior hidden,frame hidden,colbacktitle=#1,
    borderline east={1pt}{0pt}{#2},
    attach boxed title to top right={yshift=-8pt,yshifttext=-8pt},
    boxed title style={boxsep=3pt,boxrule=1pt,colframe=#2,sharp corners,left=4pt,right=4pt},
    bottom=0mm
    },
flagstyle/dl/.style 2 args={
    interior hidden,frame hidden,colbacktitle=#1,
    borderline west={1pt}{0pt}{#2},
    attach boxed title to bottom left={yshift=8pt,yshifttext=8pt},
    boxed title style={boxsep=3pt,boxrule=1pt,colframe=#2,sharp corners,left=4pt,right=4pt},
    top=0mm
    },
flagstyle/dr/.style 2 args={
    interior hidden,frame hidden,colbacktitle=#1,
    borderline east={1pt}{0pt}{#2},
    attach boxed title to bottom right={yshift=8pt,yshifttext=8pt},
    boxed title style={boxsep=3pt,boxrule=1pt,colframe=#2,sharp corners,left=4pt,right=4pt},
    top=0mm
    }
}

% Box in the shape of a filing divider, position of tab can be ul (up left), uc (up center), ur (up right), dl (down left), dc (down center) or dr (down right). Default is ul (upper left)
\NewTColorBox{filingbox}{ D(){ul} O{black} m O{} }{enhanced,
    top=1mm,bottom=1mm,left=1mm,right=1mm,
    title={#3},
    fonttitle=\sffamily\bfseries,
    coltitle=black,
    filingstyle/#1={#2!10}{#2},
    #4
}

% Box in the shape of a railing bar, position of tab can be ul (up left), uc (up center), ur (up right), dl (down left), dc (down center) or dr (down right). Default is ul (upper left)
\NewTColorBox{railingbox}{ D(){ul} O{black} m O{} }{enhanced,
    top=1mm,bottom=1mm,left=1mm,right=1mm,
    title={#3},
    fonttitle=\sffamily\bfseries,
    coltitle=black,
    railingstyle/#1={#2!10}{#2},
    #4
}

% Box in the shape of a flag, position of tab can be ul (up left), ur (up right), dl (down left) or dr (down right). Default is ul (upper left)
\NewTColorBox{flagbox}{ D(){ul} O{black} m O{} }{enhanced,breakable,
    top=1mm,bottom=1mm,left=1mm,right=1mm,
    title={#3},
    fonttitle=\sffamily\bfseries,
    coltitle=black,
    flagstyle/#1={#2!10}{#2},
    #4
}

\makeatletter
\newcommand*{\CreateSmartLargeOperator}[2]{
% Adapted from https://tex.stackexchange.com/questions/61598/new-command-with-cases-conditionals-if-thens/61600
    % Plain operator (no customization)
    \csdef{LargeOperator@#1@}{\csdef{LargeOperator@#1@Symbol}{\csuse{#1}}}
    % Operator with limits above and below symbol
    \csdef{LargeOperator@#1@l}{\csdef{LargeOperator@#1@Symbol}{\csuse{#1}\limits}}
    % Operato with limits beside symbol
    \csdef{LargeOperator@#1@n}{\csdef{LargeOperator@#1@Symbol}{\csuse{#1}\nolimits}}
    % Inline style operator
    \csdef{LargeOperator@#1@i}{\csdef{LargeOperator@#1@Symbol}{\textstyle\csuse{#1}}}
    % Display style operator
    \csdef{LargeOperator@#1@d}{\csdef{LargeOperator@#1@Symbol}{\displaystyle\csuse{#1}}}
    % Inline style operator with limits above and below symbol
    \csdef{LargeOperator@#1@il}{\csdef{LargeOperator@#1@Symbol}{\textstyle\csuse{#1}\limits}}
    % Inline style operator with limits beside symbol
    \csdef{LargeOperator@#1@in}{\csdef{LargeOperator@#1@Symbol}{\textstyle\csuse{#1}\nolimits}}
    % Display style operator with limits above and below symbol
    \csdef{LargeOperator@#1@dl}{\csdef{LargeOperator@#1@Symbol}{\displaystyle\csuse{#1}\limits}}
    % Display style operator with limits beside symbol
    \csdef{LargeOperator@#1@dn}{\csdef{LargeOperator@#1@Symbol}{\displaystyle\csuse{#1}\nolimits}}

% NOTE: In the command below, ##1 denotes the operator. It is NOT to be used as an argument!
\def\LargeOperatorSpecs@i##1,##2,##3,##4,##5,##6,##7\@nil{
% If no arguments, operate over n from 1 to infinity
    \ifx$##2$\csuse{LargeOperator@##1@Symbol}_{n=1}^{\infty}\else
    % If one argument, operate over n from ##2 to infinity
        \ifx$##3$\csuse{LargeOperator@##1@Symbol}_{n=##2}^{\infty}\else
        % If two arguments, operate over n from ##2 to ##3
            \ifx$##4$\csuse{LargeOperator@##1@Symbol}_{n=##2}^{##3}\else
            % If three arguments, operate over ##2 from ##3 to ##4
                \ifx$##5$\csuse{LargeOperator@##1@Symbol}_{##2=##3}^{##4}\else
                % If four arguments, operate over ##2 and ##3 from ##4 to ##5
                    \ifx$##6$\csuse{LargeOperator@##1@Symbol}_{##2,##3=##4}^{##5}\else
                    % If five arguments, operate over ##2, ##3 and ##4 from ##5 to ##6
                        \csuse{LargeOperator@##1@Symbol}_{##2,##3,##4=##5}^{##6}
                    \fi
                \fi
            \fi
        \fi
    \fi
}

% Flexible "smart" large operator macro with comma-separated arguments and optional argument for formatting. Default is over n from 1 to infinity. Adapted from https://tex.stackexchange.com/a/15722
\expandafter\DeclareDocumentCommand\csname#2\endcsname{ O{} m }{ % New operator macro
\bgroup % Group created to keep operator style (e.g. \limits) local
    \expandafter\ifx\csname LargeOperator@#1@##1\endcsname\relax
    \expandafter\@gobble\else
    \csname LargeOperator@#1@##1\expandafter\endcsname
    \fi
    \expandafter\LargeOperatorSpecs@i#1,##2,,,,,\@nil% % #1 stands in for the first "argument" of \LargeOperatorSpecs@i (the operator), the actual arguments are from ##2 onward
\egroup}
}
\makeatother

% Create the smart large operator #2 based on the large operator #1. For example, \CreateSmartLargeOperator{sum}{Sum} will define \Sum as the smart large operator based on \sum
% Equivalent Unicode characters are given here (but they are NOT the same as the operators)
\CreateSmartLargeOperator{sum}{Sum}             % Large: U+2211 ∑ (no small version)
\CreateSmartLargeOperator{prod}{Prod}           % Small: U+2293 ⊓, Large: U+220F ∏
\CreateSmartLargeOperator{coprod}{Coprod}       % Small: U+2294 ⊔, Large: U+2210 ∐
\CreateSmartLargeOperator{bigcap}{Capp}         % Small: U+2229 ∩, Large: U+22C2 ⋂
\CreateSmartLargeOperator{bigcup}{Cupp}         % Small: U+222A ∪, Large: U+22C3 ⋃
\CreateSmartLargeOperator{bigsqcup}{Kupp}       % Small: U+2294 ⊔, Large: U+2210 ∐
\CreateSmartLargeOperator{bigodot}{Odot}        % Small: U+2299 ⊙ (no large version)
\CreateSmartLargeOperator{bigoplus}{Oplus}      % Small: U+2295 ⊕ (no large version)
\CreateSmartLargeOperator{bigotimes}{Otimes}    % Small: U+2297 ⊗ (no large version)
\CreateSmartLargeOperator{biguplus}{Uplus}      % Small: U+228E ⊎ (no large version)
\CreateSmartLargeOperator{bigwedge}{Wedge}      % Small: U+2227 ∧, Large: U+22C0 ⋀
\CreateSmartLargeOperator{bigvee}{Vee}          % Small: U+2228 ∨, Large: U+22C1 ⋁


    % enhanced jigsaw,
    % colback=tcol_PRF!10,colframe=tcol_PRF!80!black,
    % boxrule=0pt,
    % fonttitle=\sffamily\bfseries,coltitle=black,
    % separator sign={},label separator={},
    % description font=\normalfont\sffamily,
    % description delimiters={(}{)},
    % attach title to upper,after title={.\ },
    % frame hidden,borderline west={2pt}{0pt}{tcol_PRF},
    % sharp corners,top=2pt,bottom=2pt,left=5pt,right=5pt,
    % before skip=10pt,after skip=10pt,breakable

\newtcbtheorem[number within=section,
    crefname={\color{tcol_PRF!50!black} preuve}{\color{tcol_PRF!50!black} preuves},
    Crefname={\color{tcol_PRF!50!black} Preuve}{\color{tcol_PRF!50!black} Preuves}
    ]{myproof}{Preuve}{
    % boxrule=0pt,boxsep=0pt,blanker,
    % borderline west={2pt}{0pt}{tcol_PRF},left=8pt,right=8pt,sharp corners,
    % before skip=10pt,after skip=10pt,breakable
    enhanced jigsaw,borderline west={2pt}{0pt}{tcol_PRF},
	breakable,
	colback = tcol_PRF!0,
	frame hidden,
	boxrule = 0sp,
	sharp corners,
	detach title,
	before upper = \tcbtitle\par\smallskip,
	coltitle = black,
	fonttitle = \bfseries\sffamily,
	separator sign none,
    after title={:},
    label separator={},
	segmentation style={solid, tcol_PRF},
    sharp corners,top=0pt,bottom=2pt,left=5pt,right=5pt,
    % add sunglasses emoji to end of proof
    overlay unbroken={
      \node[anchor=south east,inner sep=0pt] at (frame.south east) {\qed};
    },
    overlay first={
      \node[anchor=south east,inner sep=0pt] at (frame.south east) {\qed};
    },
    overlay middle={
      \node[anchor=south east,inner sep=0pt] at (frame.south east) {\qed};
    },
    overlay last={
      \node[anchor=south east,inner sep=0pt] at (frame.south east) {\qed};
    },
    }{}
    
\tcolorboxenvironment{proof}{boxrule=0pt,boxsep=0pt,blanker,
    borderline west={2pt}{0pt}{tcol_PRF},left=8pt,right=8pt,sharp corners,
    before skip=10pt,after skip=10pt,breakable
}
% tcbtheorem with proof
\tcolorboxenvironment{remark}{boxrule=0pt,boxsep=0pt,blanker,
    borderline west={2pt}{0pt}{tcol_REM},left=8pt,right=8pt,
    before skip=10pt,after skip=10pt,breakable
}
\tcolorboxenvironment{remarks}{boxrule=0pt,boxsep=0pt,blanker,
    borderline west={2pt}{0pt}{tcol_REM},left=8pt,right=8pt,
    before skip=10pt,after skip=10pt,breakable
}
\tcolorboxenvironment{example}{boxrule=0pt,boxsep=0pt,blanker,
    borderline west={2pt}{0pt}{tcol_EXA},left=8pt,right=8pt,sharp corners,
    before skip=10pt,after skip=10pt,breakable
}
\tcolorboxenvironment{examples}{boxrule=0pt,boxsep=0pt,blanker,
    borderline west={2pt}{0pt}{tcol_EXA},left=8pt,right=8pt,sharp corners,
    before skip=10pt,after skip=10pt,breakable
}

% align and align* environments with inline size
\newenvironment{talign}{\let\displaystyle\textstyle\align}{\endalign}
\newenvironment{talign*}{\let\displaystyle\textstyle\csname align*\endcsname}{\endalign}

% Setting the format for sections, subsections and subsubsections
\titleformat{\section}{\fontsize{24}{30}\sffamily\bfseries}{\thesection}{20pt}{#1}
\titleformat{\subsection}{\fontsize{16}{18}\sffamily\bfseries}{\thesubsection}{12pt}{#1}
\titleformat{\subsubsection}{\fontsize{10}{12}\sffamily\large\bfseries}{\thesubsubsection}{8pt}{#1}
% Setting the spacing for sections, subsections and subsubsections
% First argument is the left indent, second argument is the spacing above, third argument is the spacing below
\titlespacing*{\section}{0pt}{5pt}{5pt}
\titlespacing*{\subsection}{0pt}{5pt}{5pt}
\titlespacing*{\subsubsection}{0pt}{5pt}{5pt}

\newcommand{\Disp}{\displaystyle}
\newcommand{\qe}{\hfill\(\bigtriangledown\)}
\DeclareMathAlphabet\mathbfcal{OMS}{cmsy}{b}{n}
\setlength{\parindent}{0.2in}
\setlength{\parskip}{0pt}
\setlength{\columnseprule}{0pt}

\makeatletter
% Modify spacing above and below display equations
\g@addto@macro\normalsize{
    \setlength\abovedisplayskip{3pt}
    \setlength\belowdisplayskip{3pt}
    \setlength\abovedisplayshortskip{0pt}
    \setlength\belowdisplayshortskip{0pt}
}
\makeatother

\makeatletter






%================== Front page ==================%
\newcommand{\UE}[1]{\renewcommand{\UE}{#1}}
\newcommand{\sujet}[1]{\renewcommand{\sujet}{#1}}
\newcommand{\titre}[1]{\renewcommand{\titre}{#1}}
\newcommand{\enseignant}[1]{\renewcommand{\enseignant}{#1}}
\newcommand{\eleves}[1]{\renewcommand{\eleves}{#1}}

\newcommand{\makemargins}{
\makenomenclature
\pagestyle{fancy}
\fancyheadoffset{1cm}
\setlength{\headheight}{2cm}
\lhead{\includegraphics[scale=0.7]{logos/bonlogo.png}} %Affichage de l'image au top de la page
\rhead{\nouppercase{\leftmark}}
\rfoot{\thepage}
\cfoot{\textbf{\titre}}
\lfoot{\UE}
}


\newcommand{\makefrontpage}{
    
\begin{titlepage}
\ThisLRCornerWallPaper{1}{logos/Sceau.png}
	\centering %Centraliser le contenu
	\includegraphics[width=0.2\textwidth]{logos/logo ulb bleu.jpg}\par\vspace{1cm} %Insertion du logo
	{\scshape\LARGE Université Libre de Bruxelles \par} %Nom de l'université
	\vspace{1.5cm}%Espace de 1,5cm
	{\scshape\Large \UE \\ \sujet \\ \par} %sous-titre
	\vspace{1cm}%Espace de 1cm
    \rule{\linewidth}{0.2 mm} \\[0.4 cm]
	{\huge\bfseries \titre \par} \
    \rule{\linewidth}{0.2 mm} \\[1.5 cm]
	\vspace{1cm}%Espace de 3cm
    
	\begin{minipage}{0.5\textwidth} %Minipage pour faire deux colonnes
		\begin{flushleft} \large %Envoyer à gauche
		\emph{\textbf{Étudiants :}}\\ %Pour le titre au dessus des noms à gauche
        \eleves\\ %Remplacer pour chacun
		\end{flushleft}
	\end{minipage}
	~
	\begin{minipage}{0.4\textwidth}
		\begin{flushright} \large
		\emph{\textbf{Enseignants :}} \\
		 \enseignant \\
		\end{flushright}
	\end{minipage}\\[4cm]
    
	\vfill
	{\large \today\par} %Affichage de la date

\end{titlepage}
}

\newcommand{\maketoc}{
    \begin{tcolorbox}[enhanced,
        title=Contents,
        fonttitle=\fontsize{20}{24}\sffamily\bfseries\selectfont,
        coltitle=black,
        fontupper=\sffamily,
        %interior style={left color=tcol_CNT1!80,right color=tcol_CNT2!80},
        % frame style={left color=tcol_CNT1!60!black,right color=tcol_CNT2!60!black},
        attach boxed title to top center={yshift=10pt},
        boxed title style={frame hidden,
            % interior style={left color=tcol_CNT1,right color=tcol_CNT2},
            interior style={opacity=0},
            % frame style={left color=tcol_CNT1!60!black,right color=tcol_CNT2!60!black},
            height=24pt,bean arc
            %drop fuzzy shadow
        },
        top=2mm,bottom=2mm,left=2mm,right=2mm,
        before skip=20mm,after skip=20mm,
        drop fuzzy shadow,breakable]
    
    \makeatletter
    \@starttoc{toc}
    \makeatother
    
    \end{tcolorbox}
    \newpage
}

\newtcolorbox{convention}{
    enhanced,
    frame hidden,
    title=Conventions,
    fonttitle=\large\sffamily\bfseries\selectfont,
    interior code={
        \shade[top color=tcol_CNV2!60,bottom color=white] ([yshift=2mm]interior.north west) arc(-180:-90:2mm)--(interior.north east)--(interior.south east)--(interior.south west)--cycle;
    },
    overlay={
        \draw[tcol_CNV1!50!black,line width=0.5mm] ([xshift=2mm]frame.north west)--(frame.north east);
    },
    boxrule=0pt,
    left=2pt,
    right=2pt,
    sharp corners=north,
    attach boxed title to top left,
    boxed title style={
        interior hidden,
        left=1mm,
        right=1mm,
        frame code={
            \path[draw=tcol_CNV1!50!black,line width=0.5mm,fill=tcol_CNV1,rounded corners=2mm] ([xshift=2mm]frame.south east)--(frame.south east)--(frame.north east)--([xshift=0.25mm]frame.north west)--([xshift=0.25mm]frame.south west)--cycle;
        }
    },
    top=2mm,
    bottom=2mm,
    before skip=10mm,
    after skip=10mm
}



\usemintedstyle{vs}

\titleclass{\subsubsubsection}{straight}[\subsection]



%=============listing code================
\numberwithin{listing}{section}
\DeclareCaptionFormat{myformat}{\captionsetup{justification=centering}\vspace*{-2em}#1#2#3}
\newenvironment{code}{\captionsetup{type=listing, format=myformat}\captionsetup{justification=centering}\begin{minipage}{\linewidth}}{\end{minipage}\vspace{0.8em}}
% \newenvironment{code}{\captionsetup{type=listing, format=myformat}\captionsetup{justification=centering}}{}
\SetupFloatingEnvironment{listing}{name=Code}
%=========================================


%============== Subsubsubsection ==============
% \titleformat{\subsubsubsection}
%   {\normalfont\normalsize\bfseries}{\thesubsubsubsection}{1em}{}
\newcounter{subsubsubsection}[subsubsection]
\renewcommand\thesubsubsubsection{\thesubsubsection.\arabic{subsubsubsection}}
\renewcommand\theparagraph{\thesubsubsubsection.\arabic{paragraph}} % optional; useful if paragraphs are to be numbered
\titleformat{\subsubsubsection}
    {\fontsize{11}{10}\sffamily\bfseries}{\thesubsubsubsection}{8pt}{#1}

    
\titlespacing*{\subsubsubsection}
{0pt}{3.25ex plus 1ex minus .2ex}{1.5ex plus .2ex}

\makeatletter
\renewcommand\paragraph{\@startsection{paragraph}{5}{\z@}%
  {3.25ex \@plus1ex \@minus.2ex}%
  {-1em}%
  {\normalfont\normalsize\bfseries}}
\renewcommand\subparagraph{\@startsection{subparagraph}{6}{\parindent}%
  {3.25ex \@plus1ex \@minus .2ex}%
  {-1em}%
  {\normalfont\normalsize\bfseries}}
\def\toclevel@subsubsubsection{4}
\def\toclevel@paragraph{5}
\def\toclevel@paragraph{6}
\def\l@subsubsubsection{\@dottedtocline{4}{7em}{4em}}
\def\l@paragraph{\@dottedtocline{5}{10em}{5em}}
\def\l@subparagraph{\@dottedtocline{6}{14em}{6em}}
\makeatother

\setcounter{secnumdepth}{4}
\setcounter{tocdepth}{4}
% ======================================




\definecolor{bg}{gray}{0.95}


\DeclareTCBListing{mintedbox}{O{}m!O{}}{%
  breakable=true,
  listing engine=minted,
  listing only,
  minted language=#2,
  minted style=default,
  minted options={%
    linenos,
    gobble=0,
    breaklines=true,
    breakafter=,,
    fontsize=\small,
    numbersep=8pt,
    #1},
  boxsep=0pt,
  left skip=0pt,
  right skip=0pt,
  left=25pt,
  right=0pt,
  top=3pt,
  bottom=3pt,
  arc=5pt,
  leftrule=0pt,
  rightrule=0pt,
  bottomrule=2pt,
  toprule=2pt,
  colback=bg,
  colframe=orange!70,
  enhanced,
  overlay={%
    \begin{tcbclipinterior}
    \fill[orange!20!white] (frame.south west) rectangle ([xshift=20pt]frame.north west);
    \end{tcbclipinterior}},
  #3}


  %================================
% NOTE BOX
%================================

\usetikzlibrary{arrows,calc,shadows.blur}
\tcbuselibrary{skins}
\newtcolorbox{note}[1][]{%
	enhanced jigsaw,
	colback=gray!20!white,%
	colframe=gray!80!black,
	size=small,
	boxrule=1pt,
	title=\textbf{Note:-},
	halign title=flush center,
	coltitle=black,
	breakable,
	drop shadow=black!50!white,
	attach boxed title to top left={xshift=1cm,yshift=-\tcboxedtitleheight/2,yshifttext=-\tcboxedtitleheight/2},
	minipage boxed title=1.5cm,
	boxed title style={%
			colback=white,
			size=fbox,
			boxrule=1pt,
			boxsep=2pt,
			underlay={%
					\coordinate (dotA) at ($(interior.west) + (-0.5pt,0)$);
					\coordinate (dotB) at ($(interior.east) + (0.5pt,0)$);
					\begin{scope}
						\clip (interior.north west) rectangle ([xshift=3ex]interior.east);
						\filldraw [white, blur shadow={shadow opacity=60, shadow yshift=-.75ex}, rounded corners=2pt] (interior.north west) rectangle (interior.south east);
					\end{scope}
					\begin{scope}[gray!80!black]
						\fill (dotA) circle (2pt);
						\fill (dotB) circle (2pt);
					\end{scope}
				},
		},
	#1,
}


 


% Based on 'Fun Template 1', available at https://www.overleaf.com/latex/templates/fun-template-1/drwvdzsrpgzz
\usetikzlibrary{shapes,decorations,arrows,calc,arrows.meta,fit,positioning, backgrounds}
% A command to draw ''NIL'' inside a black box
\newcommand\Nilbox{%
  \normalsize\colorbox{black}{\textcolor{white}{\textsf{\bfseries NIL}}}}

\pgfdeclarelayer{background}
\pgfsetlayers{background,main}

\tikzset{
    % -Latex,
    arrow/.style={-Latex, , thick},
    auto,node distance =1 cm and 1 cm,semithick,
    state/.style ={ellipse, draw, minimum width = 0.7 cm},
    point/.style = {circle, draw, inner sep=0.04cm,fill,node contents={}},
    bidirected/.style={Latex-Latex,dashed},
    el/.style = {inner sep=2pt, align=left, sloped},
      NBlack/.style={
        circle,
        minimum size=33pt,
        fill=black,
        draw,
        font=\color{black}\sffamily\large\bfseries
      },
      NRed/.style={
        circle,
        minimum size=33pt,
        fill=red!90!black,
        draw,
        font=\color{black}\sffamily\large\bfseries
      },
      Normal/.style={
        circle,
        minimum size=33pt,
        draw,
        font=\color{black}\sffamily\large\bfseries
      },
      pics/solo/.style args={#1}{
        code={
        \node[Normal] 
          (#1) {#1};
        }
      },
      pics/double/.style args={#1/#2/#3}{
        code={
        \node[Normal] 
          (#3-left) {#1};
        \node[Normal,right=8pt of #3-left] 
          (#3-right) {#2};
        \draw[->,>=latex,thick] 
          (#3-left) -- (#3-right);  
        \begin{pgfonlayer}{background}
          \node[
            name=#3,
            draw,
            inner sep=8pt,
            % fill=gray!10,
            dashed,
            fit={(#3-left) (#3-right)}
          ] {};
        \end{pgfonlayer} 
        }
      },
      pics/triple/.style args={#1/#2/#3/#4}{
        code={
        \node[Normal] 
          (#4-left) {#1};
        \node[Normal,right=8pt of #4-left] 
          (#4-middle) {#2};
        \node[Normal,right=8pt of #4-middle] 
          (#4-right) {#3};
        \draw[->,>=latex,thick] 
          (#4-middle) -- (#4-left);  
        \draw[->,>=latex,thick] 
          (#4-middle) -- (#4-right);  
        \begin{pgfonlayer}{background}
          \node[
            name=#4,
            draw,
            inner sep=8pt,
            % fill=gray!10,
            dashed,
            fit={(#4-left) (#4-right)}
          ] {};
        \end{pgfonlayer} 
        }
      }
}

% Tikz settings optimized for causal graphs. (heaps)
\tikzstyle{vertex}=[draw,fill=black!15,circle,minimum size=18pt,inner sep=0pt]
\tikzstyle{redVertex}  =[draw,fill=red,     circle,minimum size=18pt,inner sep=0pt, text=black]
\tikzstyle{blackVertex}=[draw,circle,minimum size=18pt,inner sep=0pt, text=black]
\tikzstyle{nil}        =[draw=none,fill=white,circle,minimum size=18pt,inner sep=0pt, text=white]
\tikzstyle{Normal}        =[draw,fill=white,circle,minimum size=18pt,inner sep=0pt, text=black]
\tikzstyle{yellow}        =[draw,fill=yellow,circle,minimum size=18pt,inner sep=0pt, text=black]
\tikzstyle{green}        =[draw,fill=green,circle,minimum size=18pt,inner sep=0pt, text=black]

\MakeRobust{\Call}

\begin{document}

%================= Settings front page =================
\titre{Synthèse IA} %Titre du fichier .pdf
\UE{INFO-F311} %Nom de la UE
\sujet{Intelligence Artificielle} %Nom du sujet

\enseignant{T. \textsc{Lenaerts}}  %Nom des enseignants
% Use \\ TO BREAK LINE

\eleves{Rayan \textsc{Contuliano Bravo}}
% \maketitle
\makemargins %Afficher les marges
\makefrontpage
\maketoc

%=======================================================

% this is the orginal latex code of the template
% \begin{convention}
$\F$ denotes either $\R$ or $\C$.\\
$\N$ denotes the set $\{1,2,3,\ldots\}$ of natural numbers (excluding $0$).\\
Inner products are taken to be linear in the first argument and conjugate linear in the second.\\
The Einstein summation convention is used for tensors unless otherwise specified.
\end{convention}

\newpage

\section{Main structure}

We start by defining several theorem-type environments 
(e.g. definition, proposition, theorem, lemma, corollary) using the 
\verb!\newtcbtheorem! macro from the \href{https://ctan.org/pkg/tcolorbox}{\texttt{tcolorbox}} package. 
In a real mathematics document, these will all have the same format (perhaps differing only in color). 
The purpose of having different formats here is to help you choose your favorite one (
or adapt one or more of them to your liking) and apply it to all of the theorem-type environments.\\

For best practices and other general advice on writing in \LaTeX, see \cite{evanchen,gleave}.

\subsection{Theorem environments}

\begin{definition}{}{continuous}
A function $f:I\to\R$ is \textbf{continuous} at $a\in I$ if $\limx f(x)=f(a)$.
\end{definition}

\begin{definition}{The Derivative}{derivative}
The \textbf{derivative} of a function $f:I\to\R$ at $a\in I$ is given by:
\begin{equation*}
    f'(a)=\limx\frac{f(x)-f(a)}{x-a}
\end{equation*}
\end{definition}

Note that \Cref{derivative} has a name (\nameref{derivative}), while \Cref{continuous} does not.

\begin{center}
You know those awesome commutative diagrams?\\
\begin{tikzcd}
    A \arrow[r,"p"] \arrow[d,red,"q"'] & B \arrow[d,"r" red] \\
    C \arrow[r,red,"s"' blue] & D
\end{tikzcd}\\
The derivative has \emph{nothing} to do with them!
\end{center}

\begin{proposition}{Differentiable \implies\ continuous}{diffcont}
If $f$ is differentiable at $a$, then $f$ is continuous at $a$.
\end{proposition}
\begin{proof}
Exercise (but only because this is a template).
\end{proof}

The converse of \Cref{diffcont} is not true in general.

\begin{examples}\leavevmode % This is needed to start the list on the next line so it won't be misaligned
\begin{enumerate}
    \item $f(x)=\abs{x}$ \hfill (not differentiable at $0$)
    \item $f(x)=\begin{cases} \sin(x) & x\ge 0 \\ 0 & x<0 \end{cases}$ \hfill (not differentiable at $0$)
\end{enumerate}
\end{examples}

\begin{theorem}{Two statements}{twost}
The following statements are true:
\begin{enumerate}[ref={(\arabic*)}]
    \item\label{twost:1} First statement
    \item\label{twost:2} Second statement
\end{enumerate}
\end{theorem}
\begin{proof}% For some reason, the proof environment does not need \leavevmode
\begin{enumerate}
    \item Proof of \ref{twost:1}.
    \item Proof of \ref{twost:2}. Note that the QED symbol is on this line, thanks to \verb!\qedhere!. \qedhere % \qedhere is to place the qed symbol here instead of in the next line
\end{enumerate}
\end{proof}

\begin{corollary}{Another statement}{thirdst}
Third statement.
\end{corollary}
\begin{proof}
This follows by combining \ref{twost:1} and \ref{twost:2} of \Cref{twost}. $\gets$ \scriptsize{\textsf{Try clicking on these}}
\end{proof}
\begin{remark}
This corollary is also obvious.
\end{remark}

\begin{corollary}{}{}
Unlike \Cref{thirdst}, this corollary has no name.
\end{corollary}

\begin{lemma}{}{square}
$(a+b)^2=a^2+2ab+b^2$
\end{lemma}
\begin{proof}\let\qed\relax
Expand the left side. Note that there is no QED symbol in this proof, thanks to \verb!\let\qed\relax!.
\end{proof}
\begin{remarks}\leavevmode % This is needed to start the list on the next line so it won't be misaligned
\begin{enumerate}
    \item It's also kind of obvious.
    \item No extra points for guessing what $(a-b)^2$ is.
\end{enumerate}
\end{remarks}

\begin{example}
$(2+4)^2=2^2+2\cdot 2\cdot 4+4^2=4+16+16=36$
\end{example}

\begin{proposition}{}{trig_square}
\begin{equation}\label{trig_square_eqn} (\cos(x)+i\sin(x))^2=\cos(x)^2-\sin(x)^2+2i\sin(x)\cos(x)
\end{equation}
\end{proposition}
\begin{proof}
Exercise (use \Cref{square}).
\end{proof}

Note that referencing \cref{trig_square} is not the same as referencing \eqref{trig_square_eqn}.

\begin{lemma}{Euler's identity}{euler}
\begin{equation*}
    e^{ix}=\cos(x)+i\sin(x)
\end{equation*}
\end{lemma}

\begin{theorem}{}{double_angle}
\begin{align}
    \cos(2x)=\cos(x)^2-\sin(x)^2 && \sin(2x)=\sin(x)\cos(x)
\end{align}
\end{theorem}
\begin{proof}
By \Cref{euler}, the left side of \eqref{trig_square_eqn} is simply $\qty(e^{ix})^2=e^{2ix}$. Now take the real and imaginary parts of the right side.
\end{proof}

This proof assumes $x$ and $y$ are real numbers. What would you do if they were complex numbers?

\begin{corollary}{}{}
Suppose $\alpha$ and $\beta$ are the angles shown below:
\begin{center}
    \begin{minipage}{7cm}
    \begin{center}
        \begin{tikzpicture}[scale=0.6]
            % \draw (-4,0)--(0,0)--(0,3)--cycle; % Triangle with no labels
            \draw (-4,0)--(0,0) node[midway,below]{4}           % Side of length 4
                --(0,3) node[midway,right]{3}                   % Side of length 3
                --cycle node[midway,above=2pt]{5};              % Side of length 5
            \draw (-3,0) arc(0:{atan(3/4)}:1)                   % Arc denoting angle α
                node[midway,xshift=6pt,yshift=3pt]{$\alpha$};   % Label α
            \draw (-0.4,0)--(-0.4,0.4)--(0,0.4);
        \end{tikzpicture}
    \end{center}
    \end{minipage}
    \begin{minipage}{7cm}
    \begin{center}
        \begin{tikzpicture}[scale=0.25]
            % \draw (0,0)--(24,0)--(0,7)--cycle; % Triangle with no labels
            \draw (0,0)--(24,0) node[midway,below]{24}          % Side of length 24
                --(0,7) node[midway,above=2pt]{25}              % Side of length 25
                --cycle node[midway,left]{7};                   % Side of length 7
            \draw (0,5) arc(-90:{-atan(7/24)}:2)                % Arc denoting angle β
                node[midway,xshift=2pt,yshift=-7pt]{$\beta$};   % Label β
            \draw (1,0)--(1,1)--(0,1);
        \end{tikzpicture}
    \end{center}
    \end{minipage}
\end{center}
\vspace{-5pt} % Remove some of the vertical space created by \end{center}
Then $\beta=2\alpha$.
\end{corollary}
\begin{proof}
Exercise (use \Cref{double_angle}).
\end{proof}

You can generate other Pythagorean triples this way\footnote{In fact, you can generate \emph{all} Pythagorean triples this way. Can you figure out how?}.

\subsection{Other options}

There's no need to stick to the
\tcbox[colback=blue!10,colframe=blue,
    fontupper=\slshape,
    boxrule=1pt,boxsep=0pt,sharp corners,on line,
    top=2pt,bottom=2pt,left=2pt,right=2pt
]{status quo}
with \LaTeX\ formatting. You can go
\tcbox[blanker,fontupper=\sffamily\bfseries,
left=4mm,right=4mm,on line,
overlay={
    \draw[shift={([xshift=2mm]frame.west)},thin,red!70!black,fill=yellow] (0.05,0.3)--(-0.1,-0.1)--(0.02,-0.02)--(-0.05,-0.3)--(0.1,0.1)--(-0.02,0.02)--cycle;
    \draw[shift={([xshift=-2mm]frame.east)},thin,red!70!black,fill=yellow] (0.05,0.3)--(-0.1,-0.1)--(0.02,-0.02)--(-0.05,-0.3)--(0.1,0.1)--(-0.02,0.02)--cycle;}
]{WILD}
!

\begin{theorem}[colback=black,colframe=green,colupper=white,
    fontupper=\ttfamily,
    boxrule=4pt,
    interior style={left color=black,right color=black,middle color=gray!80!black}, % middle color must come AFTER left color and right color!
    frame style={top color=green,bottom color=orange,middle color=pink}, % middle color must come AFTER top color and bottom color!
    coltitle=red,boxed title style={colback=yellow,colframe=gray},
    ,overlay={
        \fill[pattern=horizontal lines,pattern color=red] ([xshift=-38mm]interior.south east)--++(8mm,0)--([yshift=30mm]interior.south east)--++(0,8mm)--cycle
        node[midway,shift={(0,-0.2*sqrt(2))},sloped,cyan,font=\sffamily\bfseries]{ENGINEERS LOVE THIS!};
    },
    breakable=false]{\normalfont\scshape Residue Theorem}{}
Suppose $\Omega$ is a simply connected open subset of $\C$, $\gamma$ is a positively oriented simple closed curve  in $\Omega$ and $f:\Omega\to\C$ is holomorphic on $\Omega$ except at the points $a_1,a_2,\ldots,a_n$ inside $\gamma$. Then:
\begin{center}
\begin{tikzpicture}[scale=1,every node/.style={transform shape}]
    \definecolor{col_int_gamma}{HTML}{0B22A3} % Color of Ω=int(γ)
    \clip (-4,-2) rectangle (4,2);
    \draw[-,thick,draw=white,rounded corners=2mm,fill=col_int_gamma] (0,-1.1)--(0.5,-1.4)--(1,-1.7)--(1.5,-1.9)--(2,-1.9)--(2.5,-1.8)--(3,-1.6)--(3.5,-1.3)--(3.6,-1)--(3.8,-0.5)--(3.7,0.1)--(3.5,0.5)--(3.2,1)--(3,1.2)--(2.5,1.5)--(2,1.7)--(1.5,1.8)--(1,1.8)--(0.5,1.7)--(0,1.5)--(-0.5,1.3)--(-1,1.2)--(-1.5,1.3)--(-2,1.6)--(-2.4,1.8)--(-2.7,1.7)--(-3,1.6)--(-3.5,1.1)--(-3.8,0.5)--(-3.8,0)--(-3.7,-0.5)--(-3.5,-0.8)--(-3.3,-1)--(-3,-1.3)--(-2.5,-1.4)--(-2.2,-1.4)--(-1.9,-1.3)--(-1.5,-1.1)--(-1,-1)--(-0.5,-1)--cycle;
    \draw[-{Stealth[length=6pt,width=4pt]}] (3.68,-0.8)--(3.7,-0.75);
    \draw[-{Stealth[length=6pt,width=4pt]}] (-0.2,1.42)--(-0.25,1.4);
    \draw[-{Stealth[length=6pt,width=4pt]}] (-3.21,-1.09)--(-3.15,-1.15);
    \fill (-2.5,1.2) circle (1.2pt) node[left]{$a_1$};
    \fill (-2,-0.8) circle (1.2pt) node[left]{$a_2$};
    \fill (1.2,-1.2) circle (1.2pt);
    \fill (2.8,-0.8) circle (1.2pt) node[right]{$a_n$};
    \fill (1.7,1.3) circle (1.2pt);
    \node[font=\Large] at (0.35,-1.65) {$\gamma$};
    \node[font=\Large] at (0,0) {$\displaystyle\oint_\gamma f(z)\,dz=2\pi i\sum_{k=1}^n \operatorname{Res}(f,a_k)$};
\end{tikzpicture}
\end{center}
\end{theorem}

A lot of nice integrals can be computed using the $\underbrace{\text{\href{http://residuetheorem.com/}{residue theorem}}}_{\textsf{Click on this link}}$, see \cite[\S5.2]{taylor}.\\
% \underbrace only works in math mode, so \text{...} is required to render text using it

\section{Macros}
\subsection{The \texttt{talign} environment}

The \verb!talign! and \verb!talign*! environments work like the \verb!align! and \verb!align*! environments, except they render equations in inline size. They have the same numbering and labeling options as \verb!align! and \verb!align*!.

\begin{verbbox}
\begin{align}
    \sum_{n=1}^2 \frac{1}{3n}=\int_0^1 x\,dx
\end{align}
\end{verbbox}

\begin{verbbox}
\begin{align*}
    \sum_{n=1}^2 \frac{1}{3n}=\int_0^1 x\,dx
\end{align*}
\end{verbbox}

\begin{verbbox}
\begin{talign}
    \sum_{n=1}^2 \frac{1}{3n}=\int_0^1 x\,dx
\end{talign}
\end{verbbox}

\begin{verbbox}
\begin{talign*}
    \sum_{n=1}^2 \frac{1}{3n}=\int_0^1 x\,dx
\end{talign*}
\end{verbbox}

\subsection{Vector and Matrix macros}

The \verb!\cvc!, \verb!\rvc! and \verb!\mat! commands can be used to typeset column vectors, row vectors and $m\times n$ matrices efficiently. The vector/matrix delimiters can also be changed, using the optional argument \texttt{p}, \texttt{b}, \texttt{B}, \texttt{v} or \texttt{V} (in accordance with \texttt{pmatrix}, \texttt{bmatrix}, \texttt{Bmatrix}, \texttt{vmatrix} or \texttt{Vmatrix} respectively from \href{https://ctan.org/pkg/amsmath}{\texttt{amsmath}}). The default is parentheses ( ). Of course, these commands can only be used in math mode.

\vspace{5pt}

\begin{multicols}{2}
\begin{verbbox}
$$\cvc{2,3,1}$$
\end{verbbox}

\begin{verbbox}
$$\rvc{5,7,6}$$
\end{verbbox}

\begin{verbbox}
$$\rvc{8,,4}$$
\end{verbbox}

\begin{verbbox}
$$\rvc{a,b_1,c^2,\&}$$
\end{verbbox}

\begin{verbbox}
$$\mat{4,8}{0,9}{6,3}$$
\end{verbbox}

\begin{verbbox}
$$\mat{1,7}{8,9,2,5}$$
\end{verbbox}
%%%%%%%%%%%%        %%%%%%%%%%%%        %%%%%%%%%%%%
\begin{verbbox}
$$\cvc[]{2,3,1}$$
\end{verbbox}

\begin{verbbox}
$$\rvc[p]{5,7,6}$$
\end{verbbox}

\begin{verbbox}
$$\rvc[b]{8,,4}$$
\end{verbbox}

\begin{verbbox}
$$\rvc[B]{a,b_1,c^2,\&}$$
\end{verbbox}

\begin{verbbox}
$$\mat[v]{4,8}{0,9}{6,3}$$
\end{verbbox}

\begin{verbbox}
$$\mat[V]{1,7}{8,9,2,5}$$
\end{verbbox}
\end{multicols}

\vspace{15pt}

{\sffamily From now on, every subsection will start on a new page (see the code for how to do this).}

\newcommand{\subsectionbreak}{\clearpage} % Start every subsection on a new page

\subsection{Box macros}

Here we provide three customizable \href{https://ctan.org/pkg/tcolorbox}{\texttt{tcolorbox}} environments that are perhaps less suited to theorem-type environments (but of course, they can still be used for these) and more suited to ``lower-level'' environments, e.g. discussions or examples one would like to explain in more detail.

\subsubsection*{The \texttt{filingbox} environment}

The \texttt{filingbox} environment is a \texttt{tcolorbox} environment that creates boxes in the shape of filing dividers. It has two optional arguments (the first delimited by parentheses, the second by brackets) and one mandatory argument.
\begin{itemize}
    \item The first optional argument specifies the position of the tab. It can be \texttt{ul} (up left), \texttt{uc} (up center), \texttt{ur} (up right), \texttt{dl} (down left), \texttt{dc} (down center) and \texttt{dr} (down right). The default is \texttt{ul}.
    \item The second optional argument specifies the color of the box. The background of the box is automatically set to this color with 10\% opacity. The default is black.
    \item The mandatory argument specifies the title.
\end{itemize}
Additional \texttt{tcolorbox} options can be specified after the title.

\begin{verbbox}
\begin{filingbox}{Title here}
This is a \textbf{filingbox}.
\end{filingbox}
\end{verbbox}

\begin{verbbox}
\begin{filingbox}(uc)[red]{Title here}
This is a \textbf{filingbox}.
\end{filingbox}
\end{verbbox}

\begin{verbbox}
\begin{filingbox}(ur){Title here}[halign=center]
This is a \textbf{filingbox}.
\end{filingbox}
\end{verbbox}

\begin{verbbox}
\begin{filingbox}(dl)[blue]{Title here}[left=20mm]
This is a \textbf{filingbox}.
\end{filingbox}
\end{verbbox}

\begin{verbbox}
\begin{filingbox}(dc){Title here}[before upper=\&]
This is a \textbf{filingbox}.
\end{filingbox}
\end{verbbox}

\begin{verbbox}
\begin{filingbox}(dr){Title here}[colupper=red]
This is a \textbf{filingbox}.
\end{filingbox}
\end{verbbox}

\subsubsection*{The \texttt{railingbox} environment}

The \texttt{railingbox} environment is a \texttt{tcolorbox} environment that creates boxes in the shape of railing bars. It is similar to the \texttt{filingbox} environment, except that it produces a box with sharp corners. It has the same argument structure as \texttt{filingbox}.

\begin{verbbox}
\begin{railingbox}{Title here}
This is a \textbf{railingbox}.
\end{railingbox}
\end{verbbox}

\begin{verbbox}
\begin{railingbox}(uc)[violet]{Title here}
This is a \textbf{railingbox}.
\end{railingbox}
\end{verbbox}

\begin{verbbox}
\begin{railingbox}(ur){Title here}[coltitle=orange]
This is a \textbf{railingbox}.
\end{railingbox}
\end{verbbox}

\begin{verbbox}
\begin{railingbox}(dl)[teal]{Title here}
This is a \textbf{railingbox}.
\end{railingbox}
\end{verbbox}

\begin{verbbox}
\begin{railingbox}(dc){Title here}[after upper=DC]
This is a \textbf{railingbox}.
\end{railingbox}
\end{verbbox}

\begin{verbbox}
\begin{railingbox}(dr)[pink]{Title here}
This is a \textbf{railingbox}.
\end{railingbox}
\end{verbbox}

\subsubsection*{The \texttt{flagbox} environment}

The \texttt{flagbox} environment is a \texttt{tcolorbox} environment that creates boxes in the shape of flags. It has the same argument structure as \texttt{filingbox}, except that the position of the tab can only be \texttt{ul}, \texttt{ur}, \texttt{dl} or \texttt{dr} (the default is still \texttt{ul}).

\begin{verbbox}
\begin{flagbox}{Title here}
This is a \textbf{flagbox}.\\

The flagpole extends to the bottom of the box.
\end{flagbox}
\end{verbbox}
\begin{verbbox}
\begin{flagbox}(ur)[orange]{Title here}
This is a \textbf{flagbox}.\\

The flagpole extends to the bottom of the box.
\end{flagbox}
\end{verbbox}
\begin{verbbox}
\begin{flagbox}(dl)[purple]{Title here}
This is a \textbf{flagbox}.\\

The flagpole extends to the top of the box.
\end{flagbox}
\end{verbbox}
\begin{verbbox}
\begin{flagbox}(dr){Title here}[fontupper=\itshape]
This is a \textbf{flagbox}.\\

The flagpole extends to the top of the box.
\end{flagbox}
\end{verbbox}

\subsection{Ball macros}

The command \verb!\solidball! produces a customizable solid pool ball. It has four arguments, two optional and two mandatory.
\begin{itemize}
    \item The first (optional) argument specifies the text in the center of the ball. The default is blank (no text). The text format can also be manually specified, e.g. with \verb!\color{white} abc!.
    \item The second (mandatory) argument specifies the color of the ball.
    \item The third (optional) argument specifies the color of the ``background'' of the ball, i.e. the area behind the text (though it will be shown even if there is no text). The default is \texttt{white}.
    \item The fourth (mandatory) argument specifies the size of the ball. If this is \texttt{1}, the diameter of the ball will be 1 cm. This argument can be specified as an integer (e.g. \texttt{2}), a floating-point number (e.g. \texttt{0.75}) or a mathematical expression (e.g. \texttt{2*3/8}).
\end{itemize}

There is also a starred version \verb!\solidball*!, where the background is not drawn. In this version, the third argument is redundant.

\begin{tcolorbox}[blanker,sidebyside,before skip=10pt,after skip=10pt]
\begin{verbbox}[righthand width=1.3cm]
\solidball{red}{1}
\end{verbbox}
\begin{verbbox}[righthand width=1.3cm]
\solidball[2]{blue}{0.8}
\end{verbbox}
\begin{verbbox}[righthand width=1.3cm]
\solidball{green}[pink]{0.6}
\end{verbbox}
\begin{verbbox}[righthand width=1.3cm]
\solidball[3a]{violet}[red]{1}
\end{verbbox}
\tcblower %%%%%%%% %%%%%%%% %%%%%%%% %%%%%%%%
\begin{verbbox}[righthand width=1.3cm]
\solidball*{red}{1}
\end{verbbox}
\begin{verbbox}[righthand width=1.3cm]
\solidball*[\$]{yellow}{4/5}
\end{verbbox}
\begin{verbbox}[righthand width=1.3cm]
\solidball*{blue}[green]{(9-3)/10}
\end{verbbox}
\begin{verbbox}[righthand width=1.3cm]
\solidball*[6Z]{gray}[blabla]{1}
\end{verbbox}
\end{tcolorbox}

The command \verb!\stripedball! produces a customizable striped pool ball. It has the same argument structure as \verb!\solidball!, as well as a starred version that omits the background (but the stripe is still drawn, so the third argument is no longer redundant).

\begin{tcolorbox}[blanker,sidebyside,before skip=10pt,after skip=10pt]
\begin{verbbox}[righthand width=1.3cm]
\stripedball{yellow}{1}
\end{verbbox}
\begin{verbbox}[righthand width=1.3cm]
\stripedball[2]{violet}{0.5}
\end{verbbox}
\begin{verbbox}[righthand width=1.3cm]
\stripedball{orange}[blue]{.75}
\end{verbbox}
\begin{verbbox}[righthand width=1.3cm]
\stripedball[$\rho$]{red}[green]{1}
\end{verbbox}
\tcblower %%%%%%%% %%%%%%%% %%%%%%%% %%%%%%%%
\begin{verbbox}[righthand width=1.3cm]
\stripedball*{green}{1}
\end{verbbox}
\begin{verbbox}[righthand width=1.3cm]
\stripedball*[V]{white}[brown]{1/2}
\end{verbbox}
\begin{verbbox}[righthand width=1.3cm]
\stripedball*{black}{sqrt(9/16)}
\end{verbbox}
\begin{verbbox}[righthand width=1.3cm]
\stripedball*[mb]{pink}[yellow]{1}
\end{verbbox}
\end{tcolorbox}

\newpage % Page break, i.e. skip to the next page

The command \verb!\poolball! is based on \verb!\solidball! and \verb!\stripedball! and produces the standard (Aramith Tournament) pool balls. It has one optional argument and two mandatory arguments.
\begin{itemize}
    \item The optional argument specifies the text in the center of the ball. If this is not specified, the value of the first mandatory argument will be used.
    \item The first mandatory argument specifies the number on the ball (the color and whether it is solid or striped are automatically determined from this). If this is \texttt{0} (or any multiple of 16), it will produce the cue ball.
    \item The second mandatory argument specifies the size of the ball (identically to \verb!\solidball! and \verb!\stripedball!).
\end{itemize}

\begin{center}
\renewcommand*{\arraystretch}{1.25} % Redefine \arraystretch for this table only
\begin{tabular}{cccc}
    \poolball{1}{1} & \poolball{2}{1} & \poolball{3}{1} & \poolball{4}{1} \\
    \verb!\poolball{17}{1}! & \verb!\poolball{2}{1}! & \verb!\poolball{3}{1}! & \verb!\poolball{4}{1}! \\[4pt]
    \poolball{5}{1} & \poolball{6}{1} & \poolball{7}{1} & \poolball{8}{1} \\
    \verb!\poolball{5}{1}! & \verb!\poolball{6}{1}! & \verb!\poolball{7}{1}! & \verb!\poolball{8}{1}! \\[4pt]
    \poolball{9}{1} & \poolball{10}{1} & \poolball{11}{1} & \poolball{12}{1} \\
    \verb!\poolball{9}{1}! & \verb!\poolball{10}{1}! & \verb!\poolball{11}{1}! & \verb!\poolball{12}{1}! \\[4pt]
    \poolball{13}{1} & \poolball{14}{1} & \poolball{15}{1} & \poolball{0}{1} \\
    \verb!\poolball{13}{1}! & \verb!\poolball{14}{1}! & \verb!\poolball{15}{1}! & \verb!\poolball{0}{1}!
\end{tabular}
\end{center}

The first argument is only defined modulo 16, so \verb!\poolball{19}{1}! and \verb!\poolball{-13}{1}! produce the same result as \verb!\poolball{3}{1}!.

\subsection{Smart Operator Macros}

The \verb!\Sum! command is a ``smart'' version of the \verb!\sum! command (provided by \LaTeX) that incorporates styles and upper and lower limits into its arguments. It is based on the options provided by \href{https://ctan.org/pkg/amsmath}{\texttt{amsmath}}.\\

The \verb!\Sum! command takes one optional argument and one set of mandatory arguments, separated by commas.

\begin{itemize}
\newcommand{\argnumber}[1]{ % This command is only defined locally, i.e. only within the itemize list
    \ifnum#1=1\texttt{\color{red!80!black}\#1}\fi
    \ifnum#1=2\texttt{\color{Orange!80!black}\#2}\fi
    \ifnum#1=3\texttt{\color{Dandelion!80!black}\#3}\fi
    \ifnum#1=4\texttt{\color{Green!80!black}\#4}\fi
    \ifnum#1=5\texttt{\color{blue!80!black}\#5}\fi
}
    \item If no mandatory argument is present, the sum is over $n$ and runs from $n=1$ to $\infty$.
    \item If one mandatory argument \argnumber{1} is present, the sum is over $n$ and runs from $n=$\argnumber{1} to $\infty$.
    \item If two mandatory arguments \argnumber{1},\argnumber{2} are present, the sum is over $n$ and runs from $n=$\argnumber{1} to \argnumber{2}.
    \item If three mandatory arguments \argnumber{1},\argnumber{2},\argnumber{3} are present, the sum is over \argnumber{1} and runs from \argnumber{1} =\argnumber{2} to \argnumber{3}.
    \item If four mandatory arguments \argnumber{1},\argnumber{2},\argnumber{3},\argnumber{4} are present, the sum is over \argnumber{1} \emph{and} \argnumber{2} and runs from \argnumber{1},\argnumber{2} =\argnumber{3} to \argnumber{4}.
    \item If five mandatory arguments \argnumber{1},\argnumber{2},\argnumber{3},\argnumber{4},\argnumber{5} are present, the sum is over \argnumber{1},\argnumber{2} \emph{and} \argnumber{3} runs from \argnumber{1},\argnumber{2},\argnumber{3} =\argnumber{4} to \argnumber{5}.
\end{itemize}
\begin{warning}
    The mandatory argument cannot be omitted! Simply writing \verb!\Sum! will lead to errors.
\end{warning}

\subsubsection*{Inline mode}

\begin{multicols}{2} % Inline versions
\tcbset{height=0.8cm} % Locally fix tcolorbox height to match boxes next to each other
\begin{verbbox}
$\Sum{}$
\end{verbbox}

\begin{verbbox}
$\Sum{3,8}$
\end{verbbox}

\begin{verbbox}
$\Sum{i,j,\Omega,\tau}$
\end{verbbox}
%%%%%%%%%%%%        %%%%%%%%%%%%        %%%%%%%%%%%%
\begin{verbbox}
$\Sum{4}$
\end{verbbox}

\begin{verbbox}
$\Sum{m,6,25}$
\end{verbbox}

\begin{verbbox}
$\Sum{\&,\$,\{,?,+}$
\end{verbbox}
\end{multicols}

\subsubsection*{Display mode}

\begin{multicols}{2} % Display versions
\tcbset{height=1.5cm} % Locally fix tcolorbox height to match boxes next to each other
\begin{verbbox}
$$\Sum{}$$
\end{verbbox}

\begin{verbbox}
$$\Sum{3,8}$$
\end{verbbox}

\begin{verbbox}
$$\Sum{i,j,\Omega,\tau}$$
\end{verbbox}
%%%%%%%%%%%%        %%%%%%%%%%%%        %%%%%%%%%%%%
\begin{verbbox}
$$\Sum{4}$$
\end{verbbox}

\begin{verbbox}
$$\Sum{m,6,25}$$
\end{verbbox}

\begin{verbbox}
$$\Sum{\&,\$,\{,?,+}$$
\end{verbbox}
\end{multicols}

The optional argument of the \verb!\Sum! command specifies the style of the sum. The standard \verb!\sum! command can be customized in the following ways:
\begin{itemize}
    \item By adding \verb!\limits! (to show the limits above and below the sum) or \verb!\nolimits! (to show the limits beside the sum). These are added \emph{after} the command \verb!\sum!.
    \item It can also be customized by adding \verb!\textstyle! (to show the sum in inline style) or \verb!\displaystyle! (to show the sum in display style). These are added \emph{before} the command \verb!\sum!.
\end{itemize}
In the smart \verb!\Sum! command, these options are incorporated into the optional argument.\\

There are eight options for the optional argument of \verb!\Sum!:
\vspace*{-10pt}
\begin{multicols}{2}
\begin{itemize}[leftmargin=0.5in,labelsep=0.3in]
    \item[\texttt{l}] Limits above and below
    \item[\texttt{n}] Limits beside
    \item[\texttt{i}] Inline style
    \item[\texttt{d}] Display style
    \item[\texttt{il}] Inline style with limits above and below
    \item[\texttt{in}] Inline style with limits beside
    \item[\texttt{dl}] Display style with limits above and below
    \item[\texttt{dn}] Display style with limits beside
\end{itemize}
\end{multicols}

The optional argument overrides any pre-defined options, i.e. \verb!\textstyle\Sum[d]{8}! is equivalent to \verb!\Sum[d]{8}!.\\

If the optional argument is empty, it will be treated as absent, i.e. \verb!\Sum[]{8}! is equivalent to \verb!\Sum{8}!.

\begin{multicols}{2}
\tcbset{height=1.5cm} % Locally fix tcolorbox height to match boxes next to each other
\begin{verbbox}
$$\Sum[l]{}$$
\end{verbbox}

\begin{verbbox}
$$\Sum[n]{}$$
\end{verbbox}

\begin{verbbox}
$$\Sum[i]{}$$
\end{verbbox}

\begin{verbbox}
$$\Sum[d]{}$$
\end{verbbox}
%%%%%%%%%%%%        %%%%%%%%%%%%        %%%%%%%%%%%%
\begin{verbbox}
$$\Sum[il]{}$$
\end{verbbox}

\begin{verbbox}
$$\Sum[in]{}$$
\end{verbbox}

\begin{verbbox}
$$\Sum[dl]{}$$
\end{verbbox}

\begin{verbbox}
$$\Sum[dn]{}$$
\end{verbbox}
\end{multicols}

All of the standard large operators have a ``smart'' version, as shown below:

\begin{center}
\renewcommand*{\arraystretch}{1.5} % Redefine \arraystretch for this table only
\begin{tabular}{ccccccc}
    Standard operator & \verb!\sum! & \verb!\prod! & \verb!\coprod! & \verb!\bigcap! & \verb!\bigcup! & \verb!\bigsqcup! \\
    Smart operator & \verb!\Sum! & \verb!\Prod! & \verb!\Coprod! & \verb!\Capp! & \verb!\Cupp! & \verb!\Kupp! \\
    Inline style & $\Sum[i]{}$ & $\Prod[i]{}$ & $\Coprod[i]{}$ & $\Capp[i]{}$ & $\Cupp[i]{}$ & $\Kupp[i]{}$ \\
    Display style & $\Sum[d]{}$ & $\Prod[d]{}$ & $\Coprod[d]{}$ & $\Capp[d]{}$ & $\Cupp[d]{}$ & $\Kupp[d]{}$ \\[12pt] % Add extra vertical space here
    Standard operator & \verb!\bigodot! & \verb!\bigoplus! & \verb!\bigotimes! & \verb!\biguplus! & \verb!\bigwedge! & \verb!\bigvee! \\
    Smart operator & \verb!\Odot! & \verb!\Oplus! & \verb!\Otimes! & \verb!\Uplus! & \verb!\Wedge! & \verb!\Vee! \\
    Inline style & $\Odot[i]{}$ & $\Oplus[i]{}$ & $\Otimes[i]{}$ & $\Uplus[i]{}$ & $\Wedge[i]{}$ & $\Vee[i]{}$ \\
    Display style & $\Odot[d]{}$ & $\Oplus[d]{}$ & $\Otimes[d]{}$ & $\Uplus[d]{}$ & $\Wedge[d]{}$ & $\Vee[d]{}$
\end{tabular}
\end{center}

However, if you want something more elaborate, such as $\Disp\sum_{n\in\Z\exc\{0\}}$, it is easier to type it out normally.\\

To create your own smart operator, add the following lines (for example) to the preamble:

\begin{verbatim}
        \DeclareMathOperator*{\TheOp}{Sym}
        \CreateSmartLargeOperator{TheOp}{SmartOp}
\end{verbatim}

Then the command \verb!\SmartOp{2}! will produce $\operatorname*{Sym}_{n=2}^\infty$ in inline mode and $\Disp\operatorname*{Sym}_{n=2}^\infty$ in display mode.\\

The first line defines \verb!\TheOp! as the operator denoted by Sym. The second line defines \verb!\SmartOp! as the smart version of this operator. Note that neither argument in the second line has a backslash, this is important. And of course, the operator only renders in math mode.

\subsection{Miscellaneous}

Some commands have been defined to convenience typing in \LaTeX. For example, the symbol $\R$ for real numbers is normally typed as \verb!\mathbb{R}!, which is rather cumbersome.

\begin{center}
    \begin{tabular}{*{5}{c}} % *{5}{c} is a shortcut for ccccc
        $\A$ & $\As$ & $\Abar$ & $\Ab$ & $\Atil$ \\
        \verb!\A! & \verb!\As! & \verb!\Abar! & \verb!\Ab! & \verb!\Atil!
    \end{tabular}
\end{center}

The above commands have been defined similarly for all letters of the alphabet, with exceptions for pre-defined commands (e.g. \verb!\S!, which renders as \S).\\

The \href{https://ctan.org/pkg/physics}{\texttt{physics}} package defines bracket-like commands (such as \verb!\abs! and \verb!\norm!) that automatically resize themselves according to their content. Here, we have defined the commands \verb!\floor! and \verb!\ceil! with the same features.

\begin{multicols}{2}
\begin{verbbox}[righthand width=3cm]
$\floor{4.1}$
\end{verbbox}
\begin{verbbox}[righthand width=3cm]
$\floor{x^2}$
\end{verbbox}
\begin{verbbox}[righthand width=3cm]
$\floor{\dfrac{9}{5}}$
\end{verbbox}
\begin{verbbox}[righthand width=3cm]
$\floor*{\dfrac{x^4}{y^7}}$
\end{verbbox}
%%%%%%%%%%%% %%%%%%%%%%%% %%%%%%%%%%%%
\begin{verbbox}[righthand width=3cm]
$\ceil{7.6}$
\end{verbbox}
\begin{verbbox}[righthand width=3cm]
$\ceil{x^5}$
\end{verbbox}
\begin{verbbox}[righthand width=3cm]
$\ceil{\dfrac{3}{7}}$
\end{verbbox}
\begin{verbbox}[righthand width=3cm]
$\ceil*{\dfrac{x^5}{y^3}}$
\end{verbbox}
\end{multicols}

The starred versions \verb!\floor*! and \verb!\ceil*! do not resize the brackets.\\

The \texttt{physics} package also defines the identical commands \verb!\ip! and \verb!\braket! for inner products. Here, we have redefined \verb!\ip! to take only one argument (which may contain commas), while \verb!\braket! is left as it is. Both commands have automatic resizing, which is suppressed with their starred versions.

\begin{multicols}{2}
\begin{verbbox}
$\ip{x,y}$
\end{verbbox}
%%%%%%%%%%%% %%%%%%%%%%%% %%%%%%%%%%%%
\begin{verbbox}
$\braket{x}{y}$
\end{verbbox}
\end{multicols}

The \href{https://ctan.org/pkg/amsmath}{\texttt{amsmath}} package provides various macros for trigonometric functions, such as $\verb!\sin!$ for sin. The \texttt{physics} package overwrites these macros and provides additional ones, all with automatic resizing of brackets (unless the package is loaded with the option \texttt{notrig}). Here we have extended this list of macros to the following (all with automatic resizing):

\begin{center}
    \begin{tabular}{*{9}{c}} % *{9}{c} is a shortcut for ccccccccc
        sin & asin & arcsin & sinh & asinh & arsinh & arcsinh & sen  & Re \\
        cos & acos & arccos & cosh & acosh & arcosh & arccosh & tg   & Im \\
        tan & atan & arctan & tanh & atanh & artanh & arctanh & ctg  & arg \\
        cot & acot & arccot & coth & acoth & arcoth & arccoth & senh & Arg \\
        sec & asec & arcsec & sech & asech & arsech & arcsech & tgh  & im \\
        csc & acsc & arccsc & csch & acsch & arcsch & arccsch & ctgh & ker
    \end{tabular}
\end{center}

\LaTeX\ provides a command \verb!\iff! which renders as $\Longleftrightarrow$ (identical to \verb!\Longleftrightarrow!). The \texttt{amsmath} package also provides two commands, \verb!\implies! which renders as $\Longrightarrow$ (identical to \verb!\Longrightarrow!), and \verb!\impliedby! which renders as $\Longleftarrow$ (identical to \verb!\Longleftarrow!).\\

Here, these three commands are redefined to produce shorter sets of arrows, $\Leftrightarrow$, $\Rightarrow$ and $\Leftarrow$, identical to \verb!\Leftrightarrow!, \verb!\Rightarrow! and \verb!\Leftarrow! respectively. The original (long) arrows can be obtained using the starred versions \verb!\iff*!, \verb!\implies*! and \verb!\impliedby*!.\\

Unlike the original commands, these can also be used in text mode, however the spacing needs to be accounted for depending on the text/math on either side of the arrows.

\begin{verbbox}
a\iff b\implies c\impliedby d
\end{verbbox}

\begin{verbbox}
a\iff* b\implies* c\impliedby* d
\end{verbbox}

\begin{verbbox}
$a\iff b\implies c\impliedby d$
\end{verbbox}

\begin{verbbox}
$a\iff* b\implies* c\impliedby* d$
\end{verbbox}

\newpage
\phantomsection % Required if hyperref is used
\addcontentsline{toc}{section}{References} % Adding bibliography to table of contents
\printbibliography % Print the bibliography

\vspace{10pt}

If you're struggling to find the \LaTeX\ command for a symbol, go to \Href{https://detexify.kirelabs.org/}{detexify.kirelabs.org}.\\

If you're looking for ideas/inspiration for your \LaTeX\ documents, check out these websites:
\begin{itemize}[leftmargin=0.5in]
    \item \Href{https://castel.dev/}{castel.dev}
    \item \Href{http://dec41.user.srcf.net/}{dec41.user.srcf.net}
    \item \Href{https://www.youtube.com/@DrTrefor}{youtube.com/@DrTrefor}
    \item \Href{https://coffeeintotheorems.com/}{coffeeintotheorems.com}
    \item \Href{https://www.physicsread.com/}{physicsread.com}
    \item \Href{https://www.latextemplates.com/}{latextemplates.com}
    \item \Href{https://ralphs16.github.io/}{ralphs16.github.io}
    \item \Href{https://github.com/vhbelvadi/}{github.com/vhbelvadi}
\end{itemize}

As well as \Href{https://tex.stackexchange.com/}{tex.stackexchange.com} and \Href{https://stackoverflow.com/}{stackoverflow.com}, where you can ask and answer questions.\\

And of course, the Overleaf gallery: \Href{https://www.overleaf.com/latex/templates/}{overleaf.com/latex/templates}.\\

\centering Have fun! \coolqed{0.32}


%================= Content =============================


% \section{Introduction} % (fold)
% \label{sec:introduction}
% 
\begin{definition}{Qu'est-ce que l'ia}{iadef}
    L'intelligence artificielle est une branche de l'informatique qui crée des systèmes 
    qui pensent de manière \textbf{rationnelle}
\end{definition}

\begin{definition}{Décisions rationnelles}{decisionsrationnelles}
    Penser de manière rationnelle signifie qu'on va se concentrer sur le \textbf{choix de décisions}
    qui \textit{maximisent la probabilité} d'atteindre un objectif donné. On va faire agir les systèmes de 
    manière \textbf{optimale}
\end{definition}
\begin{remarks}\leavevmode
\begin{enumerate}
    \item     Être rationel signifie donc \textbf{maximiser} l'utilité attendue.
    \item On définis un objectif par son \textbf{utilité}.
\end{enumerate}
\end{remarks}

\begin{definition}{Agent}{agent}
    Un agent est un système qui perçoit son environnement par des \textbf{capteurs} et agit sur celui-ci par des \textbf{effecteurs}.
\end{definition}

\begin{definition}{Agent rationnel}{agentrationnel}
    Un agent rationnel est un agent qui agit de manière à maximiser son utilité attendue. 
\end{definition}

\begin{remark}\leavevmode
    Les \textbf{capteurs}, \textbf{effecteurs} et l'\textbf{environnement} permettent à l'agent 
    de percevoir et d'agir sur le monde de manière \textbf{rationnelle}. L'\textbf{agent} est le système qui prend les décisions.
\end{remark}

\begin{definition}{Fonction agent}{funca}
    La fonction agent est une fonction qui prend en entrée une séquence de perceptions et retourne une action.
    \begin{math}
        f: \PP^* \rightarrow \SA \SP
    \end{math}
\end{definition}

\begin{example}\leavevmode
    % inserer une image
    \begin{figure}[H]
        \centering
        \includegraphics[width=0.7\textwidth]{./pictures/agent_func.png}
        \caption{Représentation fonction agent dans le jeu Tetris}
        \label{fig:agent} 
    \end{figure}
\end{example}

\begin{definition}{Programe Agent}{progagent}
    Un \textbf{programe agent} $l$ est \underline{exécuté} sur une \textbf{machine} $M$ 
    afin d'\underline{implémenter} la fonction agent $f$.
\end{definition}
\begin{remark}\leavevmode
    Les machines dans le monde réel sont \textbf{imparfaites} et \textbf{limitées} en temps et en mémoire.
\end{remark}

\begin{figure}[H]
    \begin{center}
        \includegraphics[width=0.25\textwidth]{./pictures/vac_state.png}
    \end{center}
    \caption{Etat de l'environnement de l'aspirateur}\label{fig:vac_state}
\end{figure}

% WARNING: Pas sur que ce soit utile/bonne façon de faire
\begin{example}\leavevmode
    Nous pouvons représenter un aspirateur comme un agent qui perçoit son environnement par des capteurs et agit sur celui-ci par des effecteurs.
    \begin{itemize}
        \item \textbf{Perception}: capteurs qui détectent la saleté et sa localisation dans l'espace
        \item \textbf{Action}: effecteurs qui déplacent l'aspirateur dans l'espace et aspire ou non
    \end{itemize}
    En imaginant la situation en figure \ref{fig:vac_state}, nous pouvons définir la fonction agent de l'aspirateur comme suit:
    \begin{table}[H]
        \caption{Fonction agent de l'aspirateur}\label{tab:agent_func}
        \begin{center}
            \begin{tabular}[c]{|l|l|}
                \hline
                \multicolumn{1}{|c|}{\textbf{Sequence de perception}} & 
                \multicolumn{1}{c|}{\textbf{Action}} \\
                \hline

                [A, Clean] & Right \\
                \hline
                [A, Dirty] & Suck \\
                \hline
                [B, Clean] & Left\\
                \hline
                [B, Dirty] & Suck\\
                \hline
                [A, Clean], [B, Clean] & Left\\
                \hline
                [A, Clean], [B, Dirty] & Suck\\
                \hline
                etc... & etc...\\
                \hline
            \end{tabular}
        \end{center}
    \end{table}
\end{example}

Pour que notre agent soit bien rationnel, il nous faut une manière de \textbf{mesurer} la \textbf{performance}
de celui-ci. Pour cela, nous allons définir une \textbf{fonction de performance} qui va mesurer la qualité des actions de l'agent.

\begin{example}\leavevmode
    On peut lui faire gagner des points ou bien lui en retirer en fonction d'une action
\end{example}
De cette manière, l'agent va savoir quelles actions lui permettent de \textbf{maximiser} son utilité attendue.

Afin de bien déterminer un environnement, les particularité de notre agents, il nous faut 
\textbf{avant toute chose} définir \textbf{\textcolor{red}{PEAS}}

\begin{definition}{PEAS}{peas}
    \begin{itemize}
        \item \textbf{Performance}: mesure de la qualité des actions de l'agent
        \item \textbf{Environnement}: type d'environnement dans lequel l'agent va évoluer
        \item \textbf{Actuateurs}: les effecteurs de l'agent
        \item \textbf{Sensors}: les capteurs de l'agent
    \end{itemize} 
\end{definition}

\begin{figure}[H]
    \begin{center}
        \includegraphics[width=0.25\textwidth]{./pictures/pacman.png}
    \end{center}
    \caption{Environnement Pacman}\label{fig:pacman}
\end{figure}

\begin{example}\leavevmode
    Pour l'environnement Pacman de la figure \ref{fig:pacman}, nous pouvons définir PEAS comme suit:
    \begin{itemize}
        \item \textbf{Performance}: -1/pas, +10/nourriture, +500/partie gagnées, -500/mort, +200/tuer un fantôme effrayé
        \item \textbf{Environnement}: labyrinthe \textbf{dynamique }de pacman
        \item \textbf{Actuateurs}: Haut, Bas, Gauche, Droite
        \item \textbf{Capteurs}: L'état entier visible
    \end{itemize}



\end{example}

\begin{definition}{Types d'environnement}{envtype}
    Il y a plusieur type d'environnement:

    \begin{itemize}
        \item \textbf{Mono-agent}: un seul agent
        \item \textbf{Multi-agent}: plusieurs agents qui maximisent leur \textbf{propre} tâche (coop ou concurentiel)
        \item \textbf{Déterministe}: l'état de l'env est déterminé \textbf{seulement} par les actions de l'agent
        \item \textbf{Stochastique}: l'environnement est non déterministe
        \item \textbf{Épisodique}: les actions de l'agent n'affectent pas les actions futures
        \item \textbf{Séquentiel}: les actions de l'agent affectent les actions futures
        \item \textbf{Dynamique}: l'environnement peut changer pendant que l'agent réfléchit
        \item \textbf{Statique}: l'environnement ne change pas pendant que l'agent réfléchit
        \item \textbf{Complètement observable}: les capteurs de l'agent perçoivent l'état complet de l'environnement
        \item \textbf{Partiellement observable}: les capteurs de l'agent perçoivent une partie de l'état de l'environnement
        \item \textbf{Discret}: un nombre fini d'états
        \item \textbf{Continu}: un nombre infini d'états
        \item \textbf{Connu}: l'agent connait les lois de l'environnement
    \end{itemize}
\end{definition}

Il existe plusieurs types d'agents qui répondent à des environnements plus complexes:
\begin{itemize}
    \item \textbf{Agent réflexe simple}: l'agent choisit son action en fonction de la \textbf{dernière} perception
    \item \textbf{Agent réflexe basé sur un modèle}: l'agent choisit son action en fonction de la \textbf{dernière} perception et d'un \textbf{état interne}(dépend de l'\textbf{historique} des perceptions) 
    % \item \textbf{Agent réflexe avec état}: l'agent choisit son action en fonction de la \textbf{dernière} perception et de l'\textbf{historique} des perceptions
    \item \textbf{Agent fondés sur des buts}: l'agent choisit son action en fonction de la dernière perception 
        ainsi que des infos relatives à l'objectif
    \item \textbf{Agent fondés sur l'utilité}: l'agent choisit son action en fonction de 
        sa satisfaction par rapport à l'état résultant
\end{itemize}

% \newpage
% % section Introduction (end)
%
%
%
% \section{Recherche Non-Informée} % (fold)
% \label{sec:recherche}
% 
\begin{definition}{Recherche non-informée}{uninformed-search}
    La recherche \textbf{non-informée} est une stratégie de recherche qui n'utilise \textbf{pas}
    d'information sur l'état de l'environnement afin de le \textbf{guider} pour trouver une solution.
    Elle explore simplement l'espace de recherche de manière systèmatique. En utilisant souvent 
    des \textit{algorithmes} comme \textbf{DFS}, \textbf{BFS}.
\end{definition}

\begin{remark}\leavevmode
    La recherche non-informée est utilisée quand on ne connait pas l'état de l'environnement. 
    Lorsqu'on ne peut quantifier la qualité d'un état en utilisant des \textbf{information heuristiques}
\end{remark}

\begin{definition}{Agent de Plannification}{planningagent}
    Les agents de plannification font des \textbf{hypothèses} sur les conséquences des actions entreprises
    et utilisent un \textbf{modèle} de l'environnement pour trouver un plan qui atteint son objectif.
\end{definition}

\underline{\textbf{Résolution de problèmes par la recherches}}:
\begin{enumerate}
    \item \textbf{Formulation de l'objectif}: L'agent doit avoir un objectif afin 
        de pouvoir organiser son comportement. Ca permet de limiter l'espace de recherche (\textit{actions entreprises})
    \item \textbf{Formulation du problème}: L'agent doit avoir un moyen de représenter les actions et les états 
        afin de pouvoir les manipuler
    \item \textbf{Recherche de la solution}: Avant d'agir dans le monde réel, l'agent  fait  une 
        simulation de séquences d'actions dans son modèle de l'environnement jusqu'à trouver un séquence 
        qui mène à l'objectif. C'est  la \textit{solution}
    \item \textbf{Exécution de la solution}: L'agent exécute la séquence d'actions dans le monde réel
\end{enumerate}

\begin{note}
    Un plan est une séquence d'actions qui mène à l'objectif.
\end{note}

\subsection{Problème de recherches} % (fold)
\label{sub:probleme_de_recherches}

\begin{definition}{Problème de Recherche}{searchprob}
    Un problème de recherche est défini par:
    \begin{itemize}
        \item \textbf{Ensemble d'État $S$}: Une situation dans lequel l'environnement peut être agencé
        \item \textbf{État initial $s_o$}: l'état dans lequel le problème commence
        % \item \textbf{Actions}: les actions possibles
        % \item \textbf{Transition}: la fonction qui définit les conséquences des actions
        \item \textbf{Actions $A(s)$}: les actions possibles dans l'état $s$
        \item \textbf{Modèle de Transition $Result(s, a)$}: la fonction qui définit les conséquences des actions
        \item \textbf{Solution}: Une séquence d'actions qui mène de l'état initial à l'état final
        \item \textbf{État final}: l'état que l'on veut atteindre
    \end{itemize}
\end{definition}

% Fais un graph pour illustrer la roumanie

\begin{figure}[H]
    \begin{center}
        \includegraphics[width=0.70\textwidth]{./pictures/roumanie.png}
    \end{center}
    \caption{Représentation simple de la Roumanie en graphe}\label{fig:romania}
\end{figure}



\begin{example}\leavevmode
    Voyage en Roumanie:
    \begin{itemize}
        \item \textbf{États}: les villes de Roumanie
        \item \textbf{État initial}: Arad
        \item \textbf{Actions}: les routes entre les villes adjacentes
        \item \textbf{Modèle de transition}: Atteindre une ville adjacente
        \item \textbf{Cout de l'action}: distance entre les villes
        \item \textbf{État final}: Bucharest 
    \end{itemize}
\end{example}

% subsection Probleme de recherches (end)

\subsection{Graphe d'espace d'état} % (fold)
\label{sub:graphe_d_espace_d_etat}

\begin{definition}{Graphe d'espace d'état}{stategraph}
    Un graphe d'espace d'état est un graphe qui représente les états et les actions possibles.
    \begin{itemize}
        \item \textbf{Noeuds}: les états
        \item \textbf{Arêtes}: les actions
    \end{itemize} 
    L'état initial est le noeud racine et l'état final est un noeud (ou plusieurs ?).
\end{definition}

\begin{warning}
    Dans ce genre de graphe, chaque état n'est représenté qu'\textbf{une seule fois}.
\end{warning}

\begin{remark}\leavevmode
    Il est fortement possible de ne pas pouvoir représenter un problème de recherche par un graphe d'espace d'état car 
    il y a \textbf{trop d'états} ou que les états sont \textbf{continus}.
\end{remark}

% subsection Graphe d'espace d'etat (end)

\subsection{Arbres de Recherches} % (fold)
\label{sub:arbres_de_recherches}

\begin{definition}{Arbre de Recherche}{searchtree}
    Un arbre de recherche est un arbre qui représente les états et les actions possibles.
    \begin{itemize}
        \item \textbf{Noeuds}: les états, plans pour arriver à ces états
        \item \textbf{Arêtes}: les actions
        \item \textbf{Enfants}: les états suivants (\textit{succeseurs})
    \end{itemize} 
    L'état initial est le noeud racine et l'état final est un noeud (ou plusieurs ?). 
    Les noeuds peuvent être représentés plusieurs fois, (\textbf{il est donc plus grand qu'un graphe d'espace d'état.})
\end{definition}

\begin{figure}[H]
    \begin{center}
        \includegraphics[width=0.75\textwidth]{./pictures/romst.png}
    \end{center}
    \caption{Arbre de recherche de la figure \ref{fig:romania}}\label{fig:romst}
\end{figure}

\subsubsection{Recherche dans un arbre de recherche} % (fold)
\label{sec:recherche_dans_un_arbre_de_recherche}

\begin{algorithm}[H]
    \floatname{algorithm}{Recherche dans un Arbre}
    \caption{Algorithme de recherche}\label{alg:stsearch}
    \begin{algorithmic}
        \Function{Tree-Search}{problème, stratégie} 
        \State initialise un noeud avec l'état initial du problème

        \Loop{
            \If{il ne peut plus y avoir d'état à explorer}
                \State \Return Erreur
            \EndIf
            \State choisis un noeud non exploré selon la stratégie
            \If{le noeud est l'état final}
                \State \Return le plan qui mène à l'état final
            \Else
                \State Développe le noeud et ajoute ses enfants à l'arbre
            \EndIf
        }
        \EndLoop
        \EndFunction
    \end{algorithmic}
\end{algorithm}

\begin{remarks}\leavevmode
\begin{enumerate}
    \item La frontière est l'ensemble des noeuds construits non explorés de l'arbre
    \item Pour développer un noeud de la frontiere, on le retire de la frontière et on l'ajoute à l'ensemble des noeuds explorés
        ses enfants sont ajoutés à la frontière
    \item La recherche dans un graphes est similaire à la recherche dans un arbre sauf qu'on doit vérifier si un noeud a déjà été visité
        avant de l'ajouter à la frontière
\end{enumerate}
\end{remarks}

% subsubsection Recherche dans un arbre de recherche (end)

% subsection Arbres de Recherches (end)

\begin{figure}[H]
    \begin{center}
        \includegraphics[width=0.95\textwidth]{./pictures/bfsschema.jpg}
    \end{center}
    \caption{Sens d'éxécution BFS et DFS}\label{fig:bfsschema}
\end{figure}



\subsection{DFS} % (fold)
\label{sub:dfs}

\begin{definition}{DFS}{dfs}
    La recherche en profondeur (\textbf{DFS}) est une stratégie de recherche qui explore l'arbre en allant le plus loin possible dans une branche avant de revenir en arrière. 
    \begin{itemize}
        \item \textbf{Frontière}: une pile (LIFO)
        \item \textbf{Stratégie}: on choisit le noeud le plus profond de la frontière
    \end{itemize} 
\end{definition}
% subsection DFS (end)

\subsection{BFS} % (fold) 
\label{sub:bfs} 

\begin{definition}{BFS}{bfs}
    La recherche en largeur (\textbf{BFS}) est une stratégie de recherche qui explore l'arbre en allant le plus large possible dans une branche avant de revenir en arrière. 
    \begin{itemize}
        \item \textbf{Frontière}: une file (FIFO)
        \item \textbf{Stratégie}: on choisit le noeud le moins profond de la frontière
    \end{itemize} 
\end{definition} 
% subsection BFS (end)

\textbf{DFS} est meilleur que \textbf{BFS} dans les cas suivant: 
\begin{itemize}
    \item Si il y a des limitations de mémoire 
\end{itemize}

\textbf{BFS} est meilleur que \textbf{DFS} dans les cas suivant: 
\begin{itemize}
    \item Si on veut trouver la solution la plus courte
\end{itemize}

\subsection{Iterative deepening} % (fold)
\label{sub:iterative_deepening}
\begin{note}
    L'idée est d'avoir les avantages mémoire de \textbf{DFS} et la solution optimale de \textbf{BFS}
\end{note}

\begin{definition}{Iterative deepening}{iterdeep}

    L'exploration itérative en profondeur (\textbf{Iterative deepening}) est une stratégie de 
    recherche qui explore l'arbre en faisant une recherche en profondeur avec une limite de profondeur de 
    1, puis 2, puis 3, etc. jusqu'à ce que la solution soit trouvée.


    % TODO: Se renseigner sur ça 
    % L'exploration itérative en profondeur (\textbf{Iterative deepening}) est une stratégie de recherche qui explore l'arbre en allant le plus loin possible dans une branche avant de revenir en arrière. 
    % \begin{itemize}
    %     \item \textbf{Frontière}: une pile (LIFO)
    %     \item \textbf{Stratégie}: on choisit le noeud le plus profond de la frontière
    % \end{itemize} 
\end{definition}

\begin{remark}\leavevmode
    Même si cette algorithme visite plusieurs fois les mêmes noeuds, ça n'a pas vraiment d'impact
    car le nombre de noeuds est réduit (car on mise de le trouver avant d'atteindre la limite de profondeur)
\end{remark}

% subsection Iterative deepening (end)


\subsection{UCS} % (fold) 
\label{sub:ucs} 

\begin{definition}{UCS}{ucs}
    La recherche par coût uniforme (\textbf{UCS}) est une stratégie de recherche qui explore l'arbre en allant le plus loin possible dans une branche avant de revenir en arrière. 
    \begin{itemize}
        \item \textbf{Frontière}: une file de priorité
        \item \textbf{Stratégie}: on choisit le noeud ayant le plus petit coût de la frontière
    \end{itemize} 
\end{definition} 




Pour ananlyser un algorithme, on va utilser ces différentes propriétés:
\begin{itemize}
    \item \textbf{Complet}: l'algorithme trouve toujours une solution si elle existe
    \item \textbf{Optimal}: l'algorithme trouve toujours la solution optimale (avec le plus petit coût)
    \item \textbf{Complexité en temps}: Combien de temps l'algorithme prend pour trouver une solution
    \item \textbf{Complexité en espace}: Combien de mémoire l'algorithme prend pour trouver une solution
\end{itemize}

\begin{table}[H]
    \caption{Comparaison stratégie d'exploration}\label{tab:searchcomp}
    \begin{center}
        \begin{tabular}[c]{|l||l|l|l|l|}
            \hline
            \multicolumn{1}{|c|}{\textbf{Critère}} & 
            \multicolumn{1}{|c|}{\textbf{Largeur}} &
            \multicolumn{1}{c|}{\textbf{Cout uniforme}} &
            \multicolumn{1}{c|}{\textbf{Profondeur}} &
            \multicolumn{1}{c|}{\textbf{Profondeur itérative}} \\

            \hline
            \textbf{Complet}& Oui & Oui & Non & Oui \\ 
            \hline
            \textbf{Optimal}& Oui & Oui & Non & Oui\\ 
            \hline
            \textbf{Temps}& $O(b^d)$ & $O(b^{\frac{C^*}{\epsilon}})$ & $O(b^m)$ & $O(b^d)$\\ 
            \hline
            \textbf{Espace}& $O(b^d)$ & $O(b^{\frac{C^*}{\epsilon}})$ & $O(bm)$ & $O(bd)$\\

            \hline
        \end{tabular}
    \end{center}
\end{table}

% \newpage
% % section Recherche (end)
%
%
% \section{Recherche Informée} % (fold) 
% \label{sec:recherche_informee} 
% \input{sections/3.informedSearch.tex}
% \newpage
% % section Recherche  Informée(end)
%
% \section{Recherche Locale} % (fold)
% \label{sec:recherche_locale}
% \input{sections/4.localSearch.tex}
% % section Recherche Locale (end)
%
%
% \section{Recherche Adversarial} % (fold)
% \label{sec:recherche_adversarial}
% Dans les chapitres précédents, nous avons vu comment résoudre des problèmes de recherche. 
Dans ce chapitre, nous allons nous concentrer sur les problèmes où nous devons \textbf{battre nos adversaire} 
dans des jeux à \textbf{deux ou plusieurs joueurs}.

\subsection{Définition d'un jeu} % (fold)
\label{sub:definition_d_un_jeu}

\begin{definition}{Jeu}{game}
    Un jeu est un problème de recherche avec les caractéristiques suivantes:
    \begin{itemize}
        \item \textbf{État initial, $s_0$}: la position initiale du jeu.
        \item \textbf{Joueur, $Player(s)$}: le joueur qui doit jouer.
        \item \textbf{Actions, $Action(s)$}: les coups possibles pour un joueur.
        \item \textbf{Modèle de transition, $Result(a, s)$}: L'état $s'$ qui résulte de l'action $a$ dans l'état $s$
        \item \textbf{Fonction de terminal, $isTerminal(s)$}: la fonction qui détermine si le jeu est terminé
        \item \textbf{Utilité, $Utility(s, p)$}: la fonction qui détermine le score du jeu sur les états terminaux. Qui gagne,et combien.
            $p$ est le joueur.
    \end{itemize} 
\end{definition}
\begin{note}
    Il y a plusieurs types de jeux:
    \begin{itemize}
        \item \textbf{Jeux à somme nulle}: la somme des utilités des joueurs est toujours égale à 0. Si l'un \textbf{gagne}, l'autre \textbf{perd}.
        \item \textbf{Jeux à somme général}: Les agents peuvent avoir des utilités différentes. Si l'un \textbf{gagne}, l'autre \textbf{ne perd pas forcément}. (\textit{Coop}, \textit{...})
        \item \textbf{Jeux d'équipes}: Les agents jouent en équipe contre d'autres agents.
    \end{itemize}
\end{note}
% subsection Definition d'un jeu (end)
Les algorithmes de recherche que nous avons vu précédemment ne sont pas adaptés pour les jeux car ils ne prennent pas en compte le fait que l'adversaire joue aussi. 
Nous allons donc voir des algorithmes qui vont prendre en compte le fait que l'adversaire joue aussi et vont alors recommander le meilleur coup à jouer selon toutes les possibilités.

\subsection{Minimax} % (fold)
\label{sub:minimax}

\begin{definition}{Minimax}{minimax}
    L'algorithme \textbf{Minimax} est un algorithme qui permet de trouver le meilleur coup à jouer dans un jeu \textbf{déterministe} à \textbf{somme nulle}.
    Le principe est simple, nous imaginons 2 joueurs qui jouent l'un contre l'autre, \textbf{Min} et \textbf{Max}.
    Le but de \textbf{Max} est de maximiser l'utilité et le but de \textbf{Min} est de minimiser l'utilité sachant 
    que l'autre joueur va jouer de manière optimale.
\end{definition}

\begin{figure}[H]
    \begin{center}
        \includegraphics[width=0.55\textwidth]{./pictures/minimax.png}
    \end{center}
    \caption{Minimax dans un arbre.}\label{fig:minimaxtree}
\end{figure}


L'algorithme va généré l'arbre de recherche jusqu'aux étas \textbf{terminaux} pour utiliser la fonction d'utilité. 
Il traite ensuite ces valeurs en remontant l'arbre et les choisit en fonction du joueur qui doit jouer (\textbf{Min} ou \textbf{Max})
\begin{note}
    Pour ne pas généré tous l'arbre, on peut utiliser une \textbf{profondeur limitée}.
\end{note}

Si il y a plus de 2 joueurs, nous pouvons assigner à chaque noeud un tuple de valeurs qui représente l'utilité pour chaque joueur. 
Chaque joueur va alors choisir le coup qui maximise son utilité. 

\begin{remark}\leavevmode
    Cet algortihme est basé sur \textbf{DFS}. 
    \begin{itemize}
        \item \textbf{Complexité en temps}: $O(b^m)$
        \item \textbf{Complexité en espace}: $O(bm)$
    \end{itemize}
\end{remark}

\begin{figure}[H]
    \begin{center}
        \includegraphics[width=0.65\textwidth]{./pictures/multiminimax.png}
    \end{center}
    \caption{Représentation en Multijoueur}\label{fig:mutliminimax}
\end{figure}

Cette algorithme n'est pas adapté pour les jeux où il y a beaucoup de noeuds car il va générer tout l'arbre de recherche. 
Nous allons donc voir des algorithmes qui vont couper les branches qui ne sont pas intéressantes.
\begin{algorithm}[H]
    \floatname{algorithm}{Minimax}
    \caption{Algorithme Minimax}\label{alg:minimax}
    \begin{algorithmic}
        \Function{Minimax-Search}{game, state} 
        \State player $\leftarrow$ game.\Call{To-Move}{state}
        
        \If{player = max}
            \State value, action $\leftarrow$ \Call{Max-Value}{game, state} 
        \Else 
            \State value, action $\leftarrow$ \Call{Min-Value}{game, state}
        \EndIf
        \State \Return action
        \EndFunction
        \vspace{0.5cm}
        \Function{Max-Value}{game, state}
        \If{game.\Call{Terminal-Test}{ state}}
            \State \Return game.\Call{Utility}{state, player}, $null$
        \EndIf 
        \State value, action $\leftarrow -\infty, null$

        \For{action in game.\Call{Actions}{state}}
            \State value2, action2 $\leftarrow$ \Call{Min-Value}{game, game.\Call{Result}{state, action}}
            \If{value2 $>$ value}
                \State value, action $\leftarrow$ value2, action2
            \EndIf
        \EndFor

        \Return value, action
        \EndFunction
        \vspace{0.5cm}
        \Function{Min-Value}{game, state} 
        \If{game.\Call{Terminal-Test}{ state}}
            \State \Return game.\Call{Utility}{state, player}, $null$
        \EndIf 
        \State value, action $\leftarrow \infty, null$ 

        \For{action in game.\Call{Actions}{state}}
        \State value2, action2 $\leftarrow$ \Call{Max-Value}{game, game.\Call{Result}{state, action}}%{game, game.\Call{Result}{state, action}}
            \If{value2 $<$ value}
                \State value, action $\leftarrow$ value2, action2 
            \EndIf
        % }
        \EndFor
        \State \Return value, action
        \EndFunction
    \end{algorithmic} 
\end{algorithm}
% subsection Minimax (end)
\newpage

\subsection{Alpha-Beta pruning} % (fold)
\label{sub:alpha_beta_pruning}

\begin{definition}{Alpha-Beta pruning}{alphabeta}
    L'algorithme \textbf{Alpha-Beta pruning} est un algorithme qui permet de trouver le meilleur coup à jouer dans un jeu \textbf{déterministe} à \textbf{somme nulle}.
    Il est basé sur l'algorithme \textbf{Minimax} mais il va \textbf{couper} les branches qui ne sont pas intéressantes.
    Il utilise 2 paramètres, $\alpha$ et $\beta$ qui représentent les valeurs minimales et maximales que l'on à trouvé jusqu'à présent.
    \begin{itemize}
        \item $\alpha$ est la meilleure option (valeure la plus haute) que le joueur \textbf{Max} est assuré d'obtenir. 
            \textbf{Au MOINS}.
        \item $\beta$ est la meilleure option (valeure la plus basse) que le joueur \textbf{Min} est assuré d'obtenir. 
            \textbf{Au PLUS}.
    \end{itemize} 
\end{definition}

Si le noeud est un noeud \textbf{Max}, on va mettre à jour $\alpha$ avec la valeur maximale trouvée jusqu'à présent. 
Si le noeud est un noeud \textbf{Min}, on va mettre à jour $\beta$ avec la valeur minimale trouvée jusqu'à présent. 
Tous les noeuds qui sont entre $\alpha$ et $\beta$ ne seront pas visités car ils ne sont pas intéressants.

\begin{remark}\leavevmode
    Le meilleur situation pour cette algorithme est quand les meilleurs coups sont toujours le 
    premier coup à être évalué (\textit{plus à gauche})
\end{remark}

\begin{figure}[H]
    \begin{center}
        \includegraphics[width=0.65\textwidth]{./pictures/alphabeta.png}
    \end{center}
    \caption{Alpha-Beta pruning dans un arbre}\label{fig:alphabeta} 
\end{figure}

\begin{remark}\leavevmode
    La complexité en temps est généralement plus petite que celle de minimax
    \begin{itemize}
        \item \textbf{Complexité en temps}: $O(b^{m/2})$ avec les noeuds dans le \textbf{bon} ordre
            et $O(b^{3m/4})$ si les noeuds sont visités aléatoirement.
        \item \textbf{Complexité en espace}: $O(bm)$ et $O(\sqrt{b})$ si les noeuds sont dans le bon ordre.
    \end{itemize}
\end{remark}

\begin{note}
    Même si l'on ne visite pas tous les noeuds, on a quand même trouver la solution optimale.
\end{note}

Si cet algorithme n'est pas suffisant, on peut utilisé une fonction d'évaluation qui permet d'évaluer les noeuds 
qui ne sont pas des états terminaux pour estimer leur utilité.

\scalebox{0.8}{
    \begin{minipage}{\textwidth}
        \centering
        \begin{algorithm}[H]
            \floatname{algorithm}{$\alpha-\beta$ pruning}
            \caption{Algorithme $\alpha-\beta$ pruning}\label{alg:alphabeta}
            \begin{algorithmic}
                \Function{Alpha-Beta-Search}{game, state} 
                \State player $\leftarrow$ game.\Call{To-Move}{state}
                \State value, action $\leftarrow$ \Call{Max-Value}{game, state, $-\infty$, $+\infty$}
                \State \Return action
                \EndFunction
                \vspace{0.5cm}

                \Function{Max-Value}{game, state, $\alpha$, $\beta$}
                \If{game.\Call{Terminal-Test}{ state}}
                    \State \Return game.\Call{Utility}{state, player}, $null$
                \EndIf 
                \State value, action $\leftarrow -\infty, null$

                \For{action in game.\Call{Actions}{state}}
                    \State value2, action2 $\leftarrow$ \Call{Min-Value}{game, game.\Call{Result}{state, action}, $\alpha$, $\beta$}
                    \If{value2 $>$ value}
                        \State value, action $\leftarrow$ value2, action2
                        \State $\alpha \leftarrow$ \Call{Max}{$\alpha$, value}
                    \EndIf
                    \If{value $\geq \beta$}
                        \State \Return value, action 
                    \EndIf
                \EndFor
                \Return value, action
                \EndFunction

                \vspace{0.5cm}
                \Function{Min-Value}{game, state, $\alpha$, $\beta$}
                \If{game.\Call{Terminal-Test}{ state}}
                    \State \Return game.\Call{Utility}{state, player}, $null$
                \EndIf 
                \State value, action $\leftarrow \infty, null$ 

                \For{action in game.\Call{Actions}{state}}
                \State value2, action2 $\leftarrow$ \Call{Max-Value}{game, game.\Call{Result}{state, action}, $\alpha$, $\beta$}
                    \If{value2 $<$ value}
                        \State value, action $\leftarrow$ value2, action2 
                        \State $\beta \leftarrow$ \Call{Min}{$\beta$, value}
                    \EndIf
                    \If{value $ \leq \alpha$}
                        \State \Return value, action 
                    \EndIf
                \EndFor
                \State \Return value, action
                \EndFunction
            \end{algorithmic} 
        \end{algorithm}
    \end{minipage}
}

% subsection Alpha-Beta pruning (end)

\subsection{Expectimax} % (fold)
\label{sub:expectimax}

\begin{definition}{Expectimax}{expect}
    L'algorithme Expectimax est une technique d'exploration d'arbre de décision utilisée pour la prise 
    de décision dans des environnements incertains ou probabilistes. Il est souvent appliqué dans 
    des jeux et des situations où les actions des adversaires ou des événements aléatoires peuvent influencer le déroulement du jeu.
\end{definition}

C'est donc une variante de l'algorithme \textbf{Minimax} qui va prendre en compte les noeuds de \textbf{chance}. 
Ces noeuds de chance sont des noeuds où le résultat dépend du hasard ou de l'incertitude.
La valeure d'un noeud de chance est la moyenne pondérée des valeurs de ses fils.

\begin{remark}\leavevmode
    Les noeuds de chance sont comme les noeuds du joueur \textbf{MIN} mais cette fois-ci, 
    le résultat n'est pas \textbf{\textit{certain}} (Poker, lancé de dés, mauvais move, ...).
\end{remark}

\begin{note}
    Alors que minimax représente le \textbf{pire} des cas pour un joueur (l'adversaire joue de manière optimale), 
    expectimax représente le \textbf{cas moyen} pour un joueur (l'adversaire peut aussi bien jouer de manière optimale que de manière "aléatoire").
    Expectimax permet donc de prendre des opportunités que minimax ne prendrait pas car il est trop pessimiste, restrictif.
\end{note}

\begin{algorithm}[H]
    \floatname{algorithm}{Expectimax}
    \caption{Algorithme Expectimax}\label{alg:expectimax}
    \begin{algorithmic}
        \Function{Expectimax-Search}{game, state} 
        \State player $\leftarrow$ game.\Call{To-Move}{state}
        
        \If{player = max}
            \State value, action $\leftarrow$ \Call{Max-Value}{game, state} 
        \Else 
            \State value, action $\leftarrow$ \Call{Exp-Value}{game, state}
        \EndIf
        \State \Return action
        \EndFunction
        \vspace{0.5cm}
        \Function{Max-Value}{game, state}
        \If{game.\Call{Terminal-Test}{ state}}
            \State \Return game.\Call{Utility}{state, player}, $null$
        \EndIf 
        \State value, action $\leftarrow -\infty, null$

        \For{action in game.\Call{Actions}{state}}
            \State value2, action2 $\leftarrow$ \Call{Exp-Value}{game, game.\Call{Result}{state, action}}
            \If{value2 $>$ value}
                \State value, action $\leftarrow$ value2, action2
            \EndIf
        \EndFor

        \Return value, action
        \EndFunction
        \vspace{0.5cm}
        \Function{Exp-Value}{game, state} 
        \If{game.\Call{Terminal-Test}{ state}}
            \State \Return game.\Call{Utility}{state, player}, $null$
        \EndIf 
        \State value, action $\leftarrow \infty, null$ 

        \For{action in game.\Call{Actions}{state}}
        \State p $\leftarrow$ game.\Call{Probability}{state, action}
        \State value2, action2 $\leftarrow$ \Call{Max-Value}{game, game.\Call{Result}{state, action}}%{game, game.\Call{Result}{state, action}}
        \State value $\leftarrow$ value + p * value2
        \EndFor
        \State \Return value, action
        \EndFunction
    \end{algorithmic} 
\end{algorithm}

\begin{figure}
    \begin{center}
        \includegraphics[width=0.5\textwidth]{../pictures/expectimaxdrawing.png}
    \end{center}
    \caption{Arbre Expectimax}\label{fig:expectimaxtree}
\end{figure}


\begin{note}
    Il n'est cependant pas possible d'élaguer l'arbre de recherche car pour évaluer 
    un noeud de chance, il faut évaluer tous ses fils.
\end{note}


% subsection Expectimax (end)

\subsection{Monte Carlo Tree Search (MCTS)} % (fold)
\label{sub:monte_carlo}

\begin{definition}{MCTS}{mcts}
    L'algorithme \textbf{Monte Carlo Tree Search} est un algorithme de recherche d'arbre qui utilise des simulations 
    "aléatoires" ou selon une certaine politique pour trouver la meilleure action à jouer. 
    En effet, plus on fait de simulations, plus on se rapproche de la vraie valeur de l'état. 
    Sachant cela, on va donc faire des simulations sur les noeuds qui sont les plus intéressants 
    en fonction du nombre de simulation à partir de ce noeud et de la valeur de ce noeud (\textbf{utilité}, \textbf{wins}).
    Il y a 4 étapes dans cet algorithme: 
    \begin{enumerate}
        \item \textbf{Sélection}:  L'algorithme commence par choisir un chemin dans l'arbre existant 
            en utilisant une stratégie qui combine l'exploration des nœuds prometteurs 
            et l'exploitation des nœuds déjà visités.        
        \item \textbf{Expansion}: Une fois qu'un nœud feuille est atteint, 
            l'algorithme ajoute de nouveaux nœuds pour représenter les actions possibles à partir de cet état.
        \item \textbf{Simulation}: Des simulations aléatoires sont effectuées à partir des nœuds nouvellement créés (tous ?)
            (ou des nœuds déjà existants) pour estimer la valeur de chaque action. 
            On simule  jusqu'à atteindre un état final ou une condition d'arrêt.

      \item \textbf{Backpropagation}: Les résultats des simulations sont propagés vers le haut de  
        l'arbre, mettant à jour les statistiques des nœuds visités pour refléter les nouvelles informations obtenues.
    \end{enumerate}

\end{definition}

Pour choisir le nœud à explorer, on utilise l'équation suivante: 
\begin{equation} 
    UCB(i) = \frac{w_i}{n_i} + c \sqrt{\frac{\ln N}{n_i}} 
\end{equation}
Où $w_i$ est le nombre de victoires lors des simulations à partir du noeud $i$, $n_i$ est le nombre de simulations à partir du noeud $i$,
$N$ est le nombre de simulations total du parent de $i$ et $c$ est un paramètre d'exploration qui contrôle l'importance de l'exploration par rapport à l'exploitation. 
Plus $c$ est grand, plus l'exploration est importante. 


A la fin, on choisit le noeud qui a eu le plus de simulations. En effet, 
s'il a eu beaucoup de simulations, cela veut dire qu'il est intéressant de le choisir.

\begin{note}
    Plus le nombre de simulation se rapproche de $\infty$, plus il se rapprochera du choix optimal.
\end{note}


% subsection Monte Carlo (end)



% \newpage
% % section Recherche Adversarial (end)
%
% \section{Probabilités} % (fold)
% \label{sec:probabilites}
% Les raisonnements probabilistes en \textbf{Intelligence Artificielle} sont basés sur la théorie 
des probabilités et permettent de modéliser des situations 
où l'on ne peut pas prédire avec certitude le résultat d'une action (\textit{environnements non-déterministes}).

\subsection{Quelques définitions} % (fold)
\label{sub:quelques_definitions}


\begin{definition}{Événement élémentaire}{evenementelementaire}
    Un état possible de l'environnement. 
    Il est souvent noté $\omega$.
\end{definition}

\begin{definition}{Univers}{univers}
    L'ensemble de tous les événements élémentaires possible.
    Il est souvent noté $\Omega$.
\end{definition}

\begin{definition}{Variable aléatoire}{varaleatoire}
    Une \textbf{variable aléatoire} est une fonction qui associe à chaque événement élémentaire 
    d'un espace probabilisé un nombre réel. 
    On note $A$ une variable aléatoire et $x$ une valeur prise par $A$. 
    On note $P(A=x)$ la probabilité que $A$ prenne la valeur $x$. 
    On note $P(A)$ la loi de probabilité de $A$. 
\end{definition}

\begin{remark}\leavevmode
    C'est une manière de quantifier de manière numérique les résultats d'une expérience aléatoire.

    Fonction qui prend en entrée un événeement élémentaire et qui renvoie un nombre réel (quantification).
\end{remark}

% subsection Quelques definitions (end)

\subsection{Rappel proba} % (fold)
\label{sub:rappel_proba}

\begin{definition}{Probabilité Conjointes}{probconj}
    La probabilité conjointe de deux variables aléatoires $A$ et $B$ est la probabilité de l'événement 
    où $A$ prend la valeur $x$ et $B$ prend la valeur $y$.
    \begin{align}
        P(A=x , B=y) &= P(A=x \cap B=y) \\ 
                    &= P(A=x | B=y)P(B=y) \\ 
                    &= P(B=y | A=x)P(A=x)
    \end{align}
\end{definition}

\begin{example}\leavevmode
    Rajouter des exs ?
\end{example}

\begin{definition}{Probabilité d'une disjonction}{probdisonction}
    La probabilité d'une disjonction de deux variables aléatoires $A$ et $B$ est la probabilité de l'événement 
    où $A$ prend la valeur $x$ ou $B$ prend la valeur $y$.
    \begin{equation}
        P(A=x \cup B=y) = P(A=x) + P(B=y) - P(A=x\cap B=y)
    \end{equation}
    La sousstraction est nécessaire pour éviter de compter deux fois la probabilité de l'intersection.
\end{definition}

\newpage


\begin{definition}{Probabilité Marginale}{probmarginale}
    La probabilité marginale d'une variable aléatoire $A$ est la probabilité de l'événement 
    où $A$ prend la valeur $x$.
    \begin{equation}
        P(A=x) = \sum_{y \in B} P(A=x \cap B=y)
    \end{equation} 
    Où $B$ prend toutes ses valeurs possibles.
    
\end{definition}
\begin{remark}\leavevmode
    C'est la probabilité sur un sous-ensemble de variables aléatoires.
\end{remark}


\begin{definition}{Probabilité Conditionnelle}{probconditionnelle}
    La probabilité conditionnelle de deux variables aléatoires $A$ et $B$ est la probabilité de l'événement 
    où $A$ prend la valeur $x$ sachant que $B$ prend la valeur $y$.
    \begin{equation}
        P(A=x | B=y) = \frac{P(A=x\cap B=y)}{P(B=y)}
    \end{equation} 
\end{definition}

\begin{definition}{Distribution de probabilité}{distprob}
    Une distribution de probabilité est une fonction qui associe à chaque événement élémentaire 
    d'un espace probabilisé un nombre réel positif. 
    La somme de toutes les probabilités doit être égale à 1.
    \begin{equation}
        \sum_{\omega \in \Omega} P(\omega) = 1
    \end{equation}

\end{definition}

Une distribution conditionnelle peut être vue comme une distribution \textbf{renomalisée} afin de 
respecter la contrainte de somme à 1.
Pour renormaliser une distribution, il suffit de diviser chaque probabilité par la somme de toutes les probabilités.
Cette constante de normalisation est appelée \textbf{facteur de normalisation} et est notée $\alpha$.

\begin{remark}
    Une distribution de probabilité est une variable aléatoire. 
    On peut donc parler de probabilité conjointe, marginale, conditionnelle, etc. 
    sur une distribution de probabilité. 
\end{remark}

\begin{theorem}{Bayes}{bayes}
    Ce théorème permet de calculer une probabilité conditionnelle à partir de la probabilité conditionnelle inverse. 
    \textit{inversement du conditionnement}
    \begin{equation}
        P(A|B) = \frac{P(A)P(B|A)}{P(B)}
    \end{equation}
    $P(A)$ est appelé la probabilité \textbf{a priori} de $A$. 
    $P(A|B)$ est appelé la probabilité \textbf{a posteriori} de $A$.
    A posteriori = après avoir observé $B$.
\end{theorem}

\begin{theorem}{Multiplication}{multiplication}
    Ce théorème permet de calculer une probabilité conjointe à partir de probabilités conditionnelles.
    \begin{equation}
        P(A\cap B) = P(A|B)P(B)
    \end{equation} 
\end{theorem}


\newpage


\begin{theorem}{Chainage}{chainage}
    Ce théorème permet de calculer une probabilité conjointe à partir de probabilités conditionnelles.
    \begin{align}
        P(A_1\cap A_2 \cap ... \cap A_n) &= P(A_1)P(A_2|A_1)P(A_3|A_1\cap A_2)...P(A_n|A_1\cap A_2 \cap ... \cap A_{n-1})\\
        &= \prod_{i=1}^{n} P(A_i|\bigcap_{j=1}^{i-1} A_j)
    \end{align}
    Ce théorème est une généralisation du théorème de multiplication.
\end{theorem}
\begin{remark}\leavevmode
    C'est la probabilité d'une variable aléatoire sachant que toutes les autres précédentes ont eu lieu.
\end{remark}


% subsection Rappel proba (end)
\subsection{Indépendance} % (fold)
\label{sub:independance}


\begin{definition}{Indépendance}{independance}
    Deux variables aléatoires $A$ et $B$ sont indépendantes si et seulement si 
    \begin{itemize}[label=\textbullet]
        \item $P(A | B) = P(A)$  (savoir $B$ ne change pas la probabilité de $A$) ou
        \item $P(B | A) = P(B)$ (savoir $A$ ne change pas la probabilité de $B$) ou
        \item $P(A \cap B) = P(A)P(B)$ (il n'y a pas d'impact mutuel)
    \end{itemize}
    Elle permet de simplifier les calculs de probabilités conjointes.
\end{definition}

\begin{definition}{Indépendance Conditionnelle}{independanceconditionnelle}
    Deux variables aléatoires $A$ et $B$ sont indépendantes conditionnellement à une variable aléatoire $C$ si et seulement si 
    \begin{itemize}
        \item $P(A | B \cap C) = P(A | C)$ ou 
        \item $P(A \cap B | C) = P(A | C)P(B | C)$
    \end{itemize}
    Dans ce cas, A et B sont indépendantes sachant C. C'est à dire que l'information sur la réalisation de C n'affecte pas la relation entre A et B
\end{definition}
\begin{remark}\leavevmode
    Si il n'y avait pas de condition et que $A$ et $B$ sont indépendantes, 
    on aurait $P(A | B) = P(A)$. Sauf que maintenant on a une condition $C$ qui est réalisée, et elle ne peut être supprimée
\end{remark}

% subsection Independance (end)

% \newpage
% % section Probabilites (end)

% \section{Réseaux Bayesiens} % (fold)
% \label{sec:reseaux_bayesiens}
% \begin{definition}{Réseaux Bayésiens}{bayesianNetworks}
    Un réseau bayésien est un graphe orienté acyclique (DAG) dont les nœuds représentent des variables aléatoires et les arcs représentent des dépendances conditionnelles.\\
    Ils permettent de représenter des distributions de probabilités conjointes de manière compacte, de construire des modèles de raisonnement probabiliste et de faire de l'inférence.
    \begin{itemize}
        \item Les nœuds représentent des variables aléatoires
        \item Les arcs représentent des dépendances (\textit{causilités?}) conditionnelles ainsi que des distributions de probabilités pour chaque variable aléatoire \textbf{étant donné} ses parents
        % \item Les nœuds sont associés à des probabilités conditionnelles
    \end{itemize}
    Façon compacte de représenter des probabilités conjointes.
    
\end{definition}

% \begin{example}\leavevmode
%     \begin{figure}[H]
%         \centering
%         \includegraphics[width=0.6\linewidth]{pictures/bayesianNetwork.png}
%         \caption{Réseau bayésien}
%         \label{fig:bayesianNetwork}
%     \end{figure}
% \end{example}

La topologie du \textbf{RB} modélise les relations de causalités entre les variables aléatoires. 

Un arc $X \rightarrow Y$ signifie que $X$ influence $Y$.

Si $X$ n'a pas de parents, alors sa distribution de probabilité est dite \textbf{inconditionnelle} ou \textbf{à priori}.
Si $X$ a des parents, alors sa distribution de probabilité est dite \textbf{conditionnelle} ou \textbf{à posteriori}.


\begin{example}\leavevmode
    Voici un exemple du livre \textit{Artificial Intelligence: A Modern Approach} de Stuart Russel et Peter Norvig. 
    Considérons la situation suivante: 
    \begin{itemize}
        \item Je suis au travail et mes voisins Marie et John m'ont promis de \textbf{m'appeller} chaque fois que mon \textbf{alarme} se déclenche.
        \item \textbf{Jean m'appelle} pour me dire que mon alarme s'est déclenchée. 
            \begin{itemize}
                \item Cependant, il \textbf{la confond } parfois avec la sonnerie du téléphone
            \end{itemize}
        \item \textbf{Marie m'appelle pas toujours} 
            \begin{itemize}
                \item Elle écoute de la musique et ne l'entend pas toujours
            \end{itemize}
        \item Mon alarme peut également sonner à cause de \textbf{séismes}. 
        \item $\longrightarrow$ \textbf{Comment conclure qu'il y a un cambriolage?}
    \end{itemize}
    On représente cette situation par le réseau bayésien suivant: 
    % construit le graphe avec tikz
    \begin{figure}[H]
    \centering
    \scalebox{0.8}{
        \begin{minipage}{0.45\textwidth}
            \begin{tikzpicture}
              % Dessinez les nœuds du graphe
                \node[state] (0) at (0,0) {Alarme};
                \node[state] (1) [above right =of 0] {Séisme};
                \node[state] (2) [above left =of 0] {Combriolage};
                \node[state] (3) [below right =of 0] {MarieAppelle};
                \node[state] (4) [below left =of 3] {JeanAppelle};
                
                \path (2) edge[arrow] (0);
                % \path (1) edge[arrow][loop above] (1);
                \path (1) edge[arrow] (0);
                \path (0) edge[arrow] (3);
                \path (0) edge[arrow] (4);
            \end{tikzpicture}
            \caption{Réseau bayésien de l'exemple}
            \label{fig:graph1}
        \end{minipage}
    }
    \end{figure}
    Voici les probabilités conditionnelles associées à ce réseau bayésien: 
    \begin{figure}[H]
        \centering
        \scalebox{0.8}{
            \begin{minipage}{0.45\textwidth}
                \begin{tabular}{|c|c|}
                    \hline
                    $P(Combriolage)$ & $P(\neg Combriolage)$ \\
                    \hline
                    0.001 & 0.999 \\
                    \hline
                \end{tabular}
                \caption{Probabilités de $Combriolage$}
                \label{fig:graph1}
            \end{minipage}
            \begin{minipage}{0.45\textwidth}
                \begin{tabular}{|c|c|}
                    \hline
                    $P(Séisme)$ & $P(\neg Séisme)$ \\
                    \hline
                    0.002 & 0.998 \\ 
                    \hline
                \end{tabular}
                \caption{Probabilités de $Séisme$}
                \label{fig:graph1}
            \end{minipage}
        } 

        \scalebox{0.8}{
            \begin{minipage}{0.45\textwidth}
                \begin{tabular}{|c|c|c|}
                    \hline
                    $Combriolage$ & $Séisme$ & $P(Alarme)$ \\
                    \hline
                    $V$ & $V$ & 0.95 \\ 
                    $V$ & $F$ & 0.94 \\ 
                    $F$ & $V$ & 0.29 \\ 
                    $F$ & $F$ & 0.001 \\
                    \hline
                \end{tabular}
                \caption{Probabilités conditionnelles de $Alarme$}
                \label{fig:graph1}
            \end{minipage}
        } 

        \scalebox{0.8}{
            \begin{minipage}{0.45\textwidth}
                \begin{tabular}{|c|c|}
                    \hline
                    $Alarme$ & $P(JeanAppelle)$ \\
                    \hline
                    $V$ & 0.90 \\
                    $F$ & 0.05 \\
                    \hline
                \end{tabular}
                \caption{Probabilités conditionnelles de $JeanAppelle$}
                \label{fig:graph1}
            \end{minipage}
            \begin{minipage}{0.45\textwidth}
                \begin{tabular}{|c|c|}
                    \hline
                    $Alarme$ & $P(MarieAppelle)$ \\
                    \hline
                    $V$ & 0.70 \\
                    $F$ & 0.01 \\
                    \hline
                \end{tabular}
                \caption{Probabilités conditionnelles de $MarieAppelle$}
                \label{fig:graph1}
            \end{minipage}
        } 
    \end{figure}

\end{example}

\begin{note}
    Les réseaux bayésiens peuvent avoir des variables aléatoires continues ou discrètes.
\end{note}

Nous savons que par définition, $P(A,B) = P(A|B)P(B)$.
Nous pouvons donc écrire la probabilité conjointe d'un réseau bayésien comme suit: 
\begin{equation}
    P(X_1, X_2, \dots, X_n) = \prod_{i=1}^{n} P(X_i | Parents(X_i)) 
\end{equation} 
Où $Parents(X_i)$ est l'ensemble des parents directe de $X_i$ dans le réseau bayésien.

\begin{example}\leavevmode
    En utilisant le réseau bayésien de l'exemple précédent, nous pouvons calculer la probabilité conjointe de toutes les variables aléatoires comme suit: 
    \begin{align*}
        P(C=F, S=F, A=V, J=V, M=V) \\ 
        &= P(C=F)P(S=F)P(A=V|C=F, S=F)P(J=V|A=V)P(M=V|A=V) \\
        &= 0.999 \times 0.998 \times 0.001 \times 0.90 \times 0.70 \\
        &= 0.000628
    \end{align*}
\end{example}

Pour calculer les probabilités marginales, on peut ignorer les noeuds 
\textbf{dont les descendants ne sont pas les noeuds observés}

\begin{example}\leavevmode
    \begin{align*}
        P(C=F \cap A=V) &= \sum_{s} \sum_{j} \sum_{m} P(C=F, S=s, A=V, J=j, M=m) \\
                        &= \sum_{s} P(A=V | C=f, s) P(C=F) P(S=s) 
    \end{align*}
    On peut ignorer $J$ et $M$ car ils ne sont pas des descendants de $C$ qui est observé.
    Cependant, on ne peut pas ignorer $S$ car A est un descendant de $S$ et $A$ est observé.
\end{example}


% section Reseaux Bayesiens (end)

% \section{Modèle de Markov} % (fold)
% \label{sec:modele_de_markov}
% \input{sections/8.markovmodel.tex}
% % section Modele de Markov (end)

\section{Réseaux de décisions} % (fold)
\label{sec:reseaux_de_decisions}
\input{sections/9.decisionnetwork.tex}

% section Reseaux de decisions (end)

%================= Bibliography ========================
% \newpage
% \phantomsection % Required if hyperref is used
% \addcontentsline{toc}{section}{References} % Adding bibliography to table of contents
% \printbibliography % Print the bibliography

\end{document}
